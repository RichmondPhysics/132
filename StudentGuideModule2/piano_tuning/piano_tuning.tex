\section{How to Tune a Piano}

\instructornote{%
By Matt, added to manual Spring 2017.  Time: $\sim$20 minutes

This lab is definitely off the beaten path, but some students think it's fun, and it's short.

The Wikipedia page for ``musical tuning'' includes some examples of a Bach prelude played in different tunings (``just intonation'' and ``even temperament'') and I always play those to the class.  It's really obvious how ``just temperament'' leads to a few chords that sound really badly out of tune!
}

\makelabheader %(Space for student name, etc., defined in master.tex)

\bigskip
\textbf{Apparatus}
%\vspace{-\parskip}
\begin{itemize}[nosep]
\item piano\_tuning.xlsx spreadsheet file
\end{itemize}

\medskip
\textbf{Introduction}

Tuning musical instruments is probably not something you think about every day.  But it turns out that what counts as ``in tune'' for a piano is not as straightforward as it seems.  There are lots of possible ways to tune the notes on a piano, all of which sound subtly different.

\medskip
\textbf{A C major chord}

(a) Open the spreadsheet piano\_tuning.xlsx in Excel.  Your first job is to calculate in column F the exact frequencies for the notes in a C major chord: C, E, G, and another high C. 

\vspace{-0.25in}

%\medskip
\hspace{1.10in}\raisebox{-0.25in}{130.8~Hz}

\vspace{-0.15in}
\hfill{}
Low C ($f_0$): \rule{.7in}{0.1pt}\hfill{}
E ($\frac{5}{4}f_0$): \rule{.7in}{0.1pt}\hfill{}
G ($\frac{3}{2}f_0$): \rule{.7in}{0.1pt}\hfill{}
high C ($2f_0$): \rule{.7in}{0.1pt}\hfill{}

The values you have just calculated probably sound the most ``in tune'' to your ears.

\medskip
\textbf{Just Intonation}

(b) Now you will fill in column G in the spreadsheet using a scheme of simple whole number ratios called ``just intonation.'' In this scheme, 
\begin{itemize}[nosep]
\item Each pair of notes that are seven notes away from each other on the keyboard (a ``perfect fifth'', like from C to G) has a frequency ratio of 3:2.
\item Each pair of notes that are twelve notes away from each other (an ``octave'', like from a low C to a higher C) has a frequency ratio of 2:1.  
\end{itemize}
Fill in the frequencies of column G, starting with the low C and alternating between going seven notes up and five notes down.  (For five notes down, the frequency ratio is 3:4, the same as going up by seven notes twice, then down by an octave.) For convenience, the order to calculate the frequencies is given in column H.)

(c) What is the frequency of the high C? Is it exactly twice the frequency of the low C, 130.8~Hz?
\answerspace{0.3in}

(d) Are the frequencies for the C major chord (C, E, G) exactly the same as you calculated in part (a)?
\answerspace{0.3in}

\textbf{Even Temperament}

Using the ``just intonation'' scheme you just calculated, the high C would be moved back to 261.6 Hz, which would preserve the 2:1 frequency ratio between the two C's, but would make some other pairs of notes sound \textit{awful} to our ears!  A clever alternative tuning that avoids this, called ``even temperament,'' was invented around the time of Johann Sebastian Bach.  The rule is simple:
\begin{itemize}[nosep]
\item Each pair of adjacent notes (like C to C\#) has a frequency ratio of $\sqrt[12]{2}:1$.
\end{itemize}

(e) Fill in the frequencies of column I, starting with the low C.  (The value of $\sqrt[12]{2}$ is given in cell A1.)

(f) Does the last C come out to exactly 261.6 Hz?
\answerspace{0.3in}

(g) Do the frequencies for the C major chord exactly match part (a)?  Are they at least closer than with ``just intonation?''
\answerspace{0.3in}
