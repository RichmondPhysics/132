
\section{Introduction to Cricket Graph}

\textit{CA-Cricket Graph III} is a powerful and easy-to-use application
for graphing, fitting, and manipulating data. In this appendix, we
will briefly describe how to start \textit{Cricket Graph} and how
to use the on-line help to obtain information about commands and features.

\textbf{Starting Cricket Graph }

To start the application, select the \textit{CA-Cricket Graph III}
icon from the \textbf{Apple} menu. A new, untitled data window appears
where you can enter data. Other new data windows can be created by
choosing \textbf{New} from the \textbf{File} menu. New data windows
appear as Data \#1, Data \#2, and so on until you save and name them.

\textbf{Using On-Line Help} 

On-line help provides immediate information about using key tools,
commands, and features without exiting Graph III. To use on-line help: 

\begin{enumerate}
\item Choose \textbf{CA-Cricket Graph III Help...} from the \textbf{Apple}
menu. The Help window appears. On the left side of the window is a
list of topics. 
\item To view the information for a particular topic, click the topic's
name. The help information is displayed on the right side of the window. 
\item To view another topic, click the topic's name. 
\item To see additional topics in the list, use the scroll bar. 
\item To see the previous topic in the list, click the topic's name, or
click the Previous button (up arrow). 
\item To see the next topic, click the topic's name, or the Next button
(down arrow).
\end{enumerate}
If you can't locate a particular topic, you can look for a key word
or phrase using the Search feature as follows: 

\begin{enumerate}
\item Click the Search button (magnifying glass) in the Help window. The
Search For dialog box appears. 
\item Enter the word or string you want to search for in the space provided. 
\item If you want the search to be case-sensitive (upper- or lower-case
only, or a combination), click the Match Case check box. 
\item Click the OK button to begin the search. When the first occurrence
of the word or string is found, it is highlighted. 
\item To find the next occurrence of the word or string, hold down the Option
key and click the Search button (the Search For dialog box doesn't
appear). The next occurrence of the word or string is highlighted
in the Help window. 
\item When you're finished, click the close box.\end{enumerate}

