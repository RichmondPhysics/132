
\section{Refraction of Light}

Name \rule{2.0in}{0.1pt}\hfill{}Section \rule{1.0in}{0.1pt}\hfill{}Date
\rule{1.0in}{0.1pt}

\textbf{Objective}

To investigate the path traveled by light through a plate of plexiglass
(a transparent solid material).

\textbf{Introduction}

The speed of light depends on the medium in which it travels. In passing
from one medium, at least some light energy is reflected. If the second
medium is transparent, most of the light will pass into and through
it. If the beam is not perpendicular to the boundary between the two
media, it will bend as it enters, an effect known as refraction. The
direction a single ray of light travels when refracted is given by
Snell's law:

\begin{displaymath} \frac{sin~i}{sin~r} = \frac{v_1}{v_2} = \frac{n_2}{n_1} \end{displaymath}

where

\begin{quote}
i = incident angle

r = refracted angle

v\( _{1} \) = light speed in medium 1

v\( _{2} \) = light speed in medium 1

n\( _{1} \) = index of refraction of medium 1 

n\( _{2} \) = index of refraction of medium 2
\end{quote}
\textbf{Note}: All angles are measured from the normal to the boundary
at the point the ray enters the medium. The index of refraction is
the ratio of light's speed in a vacuum, $c$, to its speed in the medium,
$v$: $n = c / v$. It is worth remembering that $n_{air} \approx 1.00$.

\vspace{0.3cm}
{\centering \includegraphics{refraction_of_light_fig_1.eps} \par}
\vspace{0.3cm}

\vspace{15mm}
\textbf{Apparatus} 

\begin{itemize}
\item light fence 
\item plexiglass block 
\item white paper, pins, and wood board 
\item protractor
\end{itemize}
\textbf{Activity} 

\begin{enumerate}
\item Put a plexiglass plate at the center of a piece of paper. Outline
its position. Identify a normal, $N_1$, perpendicular to an edge of
the plate.
\item Arrange the light source apparatus so that the parallel rays of light
cross the paper and are incident at a $30^\circ$-$35^\circ$ angle
to the normal. Trace one of these rays.
\item Sight the corresponding ray as it emerges from the other side of the
plexiglass. Trace this ray.
\item Construct the normal, $N_2$.
\item Measure and record $i$, $r$, $i'$, and $r'$. Compute and record $n_{plexiglass}$.\vspace{20mm}

\item Repeat the above procedure for different incident angles of between
$25^\circ$ and $40^\circ$.
\item Calculate and record an average $n_{plexiglass}$.\vspace{20mm}

\item Does $i = i'$? Explain.\vspace{15mm}

\item Does $r = r'$? Explain.\vspace{15mm}

\item Are the incident and exit rays parallel? Explain.\vspace{15mm}

\item What is the speed of light in the plexiglass?\vspace{15mm}

\item Under what conditions would a refraction angle be greater than an
incident angle?\vspace{15mm}
\end{enumerate}

