
\setcounter{equation}{0}
\setcounter{figure}{0}

\section{The Optical Spectrum of Hydrogen}
%Test comment by Matt

Name \rule{2.0in}{0.1pt}\hfill{}Section \rule{1.0in}{0.1pt}\hfill{}Date
\rule{1.0in}{0.1pt}


\textbf{Objective}

\begin{itemize}

\item To determine the wavelengths of the visible lines in the hydrogen spectrum using         
a spectrometer and a diffraction grating.

\item To determine the value of Rydberg's constant.

\item To compare the predicted energy levels with the measured ones.

\end{itemize}

\textbf{Introduction}

\bigskip
 
The spectral lines of the hydrogen spectrum that fall in the visible region are designated as 
the $H_\alpha$, 
$H_\beta$, $H_\gamma$, and $H_\delta$  lines.  
All (there happen to be four of them) belong to the Balmer series.  
In general, the spectrum of hydrogen can be represented by Rydberg's formula:

\begin{equation}
\frac{1}{\lambda} = R_H \left ( \frac{1}{n_f^2} - \frac{1}{n_i^2} \right )
\end{equation}

\noindent where $n_f$ can be any positive integer and $n_i$ 
takes on the values of $n_f + 1$, $n_f + 2$, 
$nf + 3$, and so on and $R_H$ is the Rydberg constant 
for hydrogen and equals $1.097 \times 10^7 m^{-1}$.

%If one writes equation 1 twice--once, say for the $H_\alpha$ wavelength $\lambda_\alpha$, 
%and once for the $H_\beta$ wavelength, $\lambda_\beta$, then one can eliminate $R_H$:
%\begin{equation}
%{1 \over {n_\beta^2}} = {1 \over n_f^2} - \left ( {\lambda_\alpha \over \lambda_\beta} \right )
%   \left ( {1 \over n_f^2} - {1 \over n_\alpha^2} \right )
%\end{equation}
%Thus, once one finds $\lambda_\alpha$ and $\lambda_\beta$ so through trial and error one can
%determine the value of the three $n$'s in equation 2 (recall they all must be integers and 
%($n_f < n_\alpha < n_\beta$).

\textbf{Apparatus}

\begin{itemize}

\item spectrometer

\item diffraction grating

\item hydrogen gas spectral tube

\item power supply for hydrogen tube

\end{itemize}


\vspace{0.2in}

\textbf{Activity 1: Measuring Spectral Lines}

\bigskip

\noindent Diffraction grating spacing $d ~ = ~\qquad\qquad\qquad${\AA}
 
Using the spectrometer, measure the angular position of the direct beam, 
$\theta_{o}$, and record it here:  $\theta_{o} ~ = ~\qquad$degrees.
For each spectral line, measure the angular position of the first order maximum 
(once on each side) and record them in columns 2 and 3 below. Then determine the
 diffraction angle on each side (columns 4 and 5) and an average of the two 
(column 6). Calculate the wavelength of each line using the relation:
\begin{equation}
\lambda = d \sin \theta
\end{equation}
where $d$ is the diffraction grating spacing.

\vspace{0.25in}

\begin{center}
\begin{tabular}{|c|c|c|c|c|c|c|c|}\hline
Line        & $\theta_{left}$     & $\theta_{right}$     & $\theta_{o} - \theta_{left}$     & $\theta_{right} - \theta_{o}$    & $\theta_{average}$ & Wavelength  &  Color \\ 
            & (degrees)  &  (degrees)  & (degrees)  & (degrees) & (degrees) &({\AA})     &        \\ \hline
$H_\alpha$  &                     &                      &                    &             &     &    & red     \\ \hline
$H_\beta$   &                     &                      &                    &             &     &    & blue  \\ \hline
$H_\gamma$  &                     &                      &                    &             &     &    & violet  \\ \hline
\end{tabular}
\end{center}

\vspace{0.5in}

\textbf{Activity 2: Calculating the Rydberg Constant}

 
%Using pairs of measured wavelengths and guesses for $n_f$ and one of the $n_i$'s, 
%calculate the other $n_i$ in equation 2.  
%When this calculated number is close to an integer you may have the correct value for the other $n$'s.
%Once you have determined the proper $n$'s,
The visible spectral lines for hydrogen correspond to transitions where $n_f$ = 2.  Using values for $n_i$ of 3, 4, and 5 for the three lines you have measured, and the wavelengths you have determined, calculate a value of $R_H$ for each line (from equation 1) and compare the average of these with the accepted value.  
Use your results to predict the wavelength of the next line in the series $\lambda_\delta$ (using $n_i$ = 6). It's measured value is $4101${\AA}.  
How does your prediction compare?     

\noindent $R_\alpha ~ = ~\qquad\qquad\qquad\qquad\qquad$

\vspace{0.4in}

\noindent $R_\beta ~ = ~\qquad\qquad\qquad\qquad\qquad$

\vspace{0.4in}

\noindent $R_\gamma ~ =~ \qquad\qquad\qquad\qquad\qquad$

\vspace{0.4in}

\noindent $R_{average} ~ =~ \qquad\qquad\qquad\qquad$  \% difference = 

\vspace{0.4in}

\noindent $\lambda_\delta ~ = ~\qquad\qquad\qquad\qquad\qquad$  \% difference =

\vspace{0.2in}


Collect values of $R_{average}$ from the other groups in the class and calculate 
an average and standard deviation. Record them below.
How does this result compare with the accepted value?
How does it compare with your individual measurement (i.e. does your 
measurement fall within the range indicated by the average and standard 
deviation)?

\vspace{1.0in}

\textbf{Activity 3: The Hydrogen Energy Level Diagram}

 
The energy levels of hydrogen can be described by the equation
\begin{equation}
E_n = -\frac{13.6~eV}{n^2}
\end{equation}
where $n$ is called the principal quantum number.
The relationship between the wavelength of the emitted light and its energy
is $E=hc/\lambda$ where $c$ is the speed of light and $h$ is Planck's constant.
Continue on next page.

\eject

Make an energy level diagram showing the transitions you believe you have measured. Calculate the transition energies (experimental) from $E=hc/\lambda$, based on your measured wavelengths.

\vspace{3.0in}

Calculate the theoretical values of these transition energies predicted by equation 3, noting that $E = E_i - E_f$.

\vspace{3.0in}

How do your measured transition energies compare with the predicted ones?

