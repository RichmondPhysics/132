\section{Magnetic Forces on Charged Particles}

\makelabheader %(Space for student name, etc., defined in master.tex)

\bigskip
\textbf{Apparatus} 
\begin{itemize}[nosep]
\item Oscilloscope
\item Bar magnet
\end{itemize}

\bigskip
\textbf{Activity 1: Magnetic Forces on Moving Charges }

An oscilloscope is built around the principle of the cathode ray tube, in which electrons emitted from a source at the back are accelerated by a series of electrodes and focused to strike a fluorescent screen at its front. The result is a visible spot that can be steered vertically and horizontally to indicate voltage as a function of time.

\begin{enumerate}[labparts]
\item \textbf{Predictions}: What, if anything, will happen to the spot on
the screen if the north pole of a magnet is brought near the left
side of the oscilloscope? What will happen if you do the same with
the south pole? What about when each of the poles are brought near
to the top? {[}Please do not touch the oscilloscope with the magnet.{]} 
\answerspace{30mm}

Turn on the oscilloscope by pressing the power button. Turn the TIME/DIV 
knob completely counterclockwise. Adjust the INTEN (intensity) and FOCUS knobs 
so that a small bright spot is formed on the oscilloscope screen by the beam of 
electrons traveling toward the screen. Do not make the spot very bright. 
%Adjust the ILLUM (illumination) knob so that the grid on the screen can be seen clearly. 
Use the horizontal and vertical POSITION controls to center the spot 
on the screen.

\item Bring the N-pole of a horizontal bar magnet near, but not touching,
the left side of the oscilloscope case at the height of the spot.
Record the direction of any deflection. Repeat with the S-pole.
\answerspace{20mm}

\item Bring the N-pole of a vertical bar magnet near, but not touching,
the top of the oscilloscope case just above the spot. Record the direction
of any deflection. Repeat with the S-pole.
\answerspace{25mm}

\item Turn off the oscilloscope.
\item Did the directions of deflections meet your expectations? Explain. 
\answerspace{25mm}

\end{enumerate}

