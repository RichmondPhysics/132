\section{Magnetic Field of the Earth}

\makelabheader %(Space for student name, etc., defined in master.tex)

\medskip
\textbf{Apparatus}
\begin{itemize}[nosep]
\item tangent galvanometer 
\item power supply 
\item digital multimeter (DMM)
\item banana plug leads (1 red, 2 black) with alligator clips
%\item switch
\item compass
\item ruler 
\item dip angle compass 
%\item wooden stand
\end{itemize}

\medskip
\textbf{Introduction} 

The magnetic field lines of a bar magnet, though emanating from its
full length, are densest near the poles. These lines are typically
not perpendicular to the faces of the magnet. The Earth, as you know,
is like a giant bar magnet, with its magnetic south pole near its
geographic north pole.

The field lines of the Earth's magnetic field, therefore, tend to
point obliquely at any given spot on its surface. The angle (let's
call it \( \phi  \)) that a line makes with a surface is known as
its \emph{dip angle} (see figure). Thus, each locality on earth has
its characteristic value for \( \phi  \) with regard to the terrestrial
magnetic field.
\begin{center} \begin{picture}(200,70) \put(0,0){\line(1,0){200}} \put(150,50){\vector(-2,-1){100}} \put(150,50){\vector(0,-1){50}} \put(150,50){\vector(-1,0){100}} \put(100,60){\makebox(0,0){$B_h$}} \put(160,25){\makebox(0,0){$B_v$}} \put(90,30){\makebox(0,0){$\bf B$}} \put(75,6){\makebox(0,0){$\phi$}} \put(60,0){\oval(10,10)[tr]} \end{picture} \end{center}
The horizontal component of the Earth's magnetic field,
\begin{equation}
B_h = B\cos\phi,
\end{equation}
causes the magnetized needle of a compass to align in the geographic
north-south direction.

You will use a tangent galvanometer to produce an additional horizontal
magnetic
field perpendicular to the Earth's field (in other words, one that
points east or west).
A tangent galvanometer consists of a vertical, circular coil of wire
with $N$ turns, a pedestal compass at the center of the coil, and electrical
contacts so that a direct current can be established in the coil.
The magnetic field at the center of the galvanometer coil caused by a current
$I$ in the coil is
\begin{equation}
B_c = N\frac{\mu_0I}{2R},
\end{equation}
where $R$ is the radius of the coil and $\mu _{0}$ is the permeability
of free space ($1.25664 \times 10^{-6}$~T$\cdot$m/A).  $B_c$
is perpendicular to the vertical plane of the coil. If the plane of
the coil is placed in the north-south direction (that is, parallel
to the compass needle when there is no current in the coil), then,
when there is current in the coil, the compass will be influenced
by two perpendicular fields, $B_h$ and $B_c$. 
The compass needle will now point in the direction of the vector
sum ${\bf B}_h+{\bf B}_c$.  

By measuring the angle by which the compass needle rotates, we can determine $B_h$.  
If we then measure the dip angle $\phi$, we can determine
the Earth's magnetic field from equation (1) above.

%\newpage
\pagebreak

\textbf{Activity}

You can record all of your measurements from this activity on the Data Sheet on the next page.
\begin{enumerate}[labparts]
\item Measure and record on the accompanying data sheet the diameter $D$ of
the galvanometer coil. Calculate and record the radius $R$ of the coil.  

\item Count and record the number of turns $N$ of the coil.

\item With the power supply off, connect its positive and negative terminals 
to the tangent galvanometer and  a DC ammeter (in series). 
Be sure the ammeter is connected in \textit{series} with 
the coil. Use long wires to connect the outside terminals of the coil
to the circuit so that the coil may be removed from the magnetic effects
of the ammeter and power supply. Be sure no magnetic material other
than the compass is in the vicinity of the coil. Turn the coarse current 
control on the power supply all the way up and the voltage controls all the 
way down.

\item Place the compass on the platform of the tangent galvanometer and rotate  
the galvanometer until the compass needle is aligned with the plane of the wire 
coil of the galvanometer. Rotate the compass body so that the needle points 
north and south on the compass.

\item Turn on the power supply and slowly turn up the fine voltage control 
until the current is about 0.2~A as measured with the meter. Record the current and calculate $B_c$ from equation (2).

\item Note that when you turned on the current, the compass needle rotates by some angle $\theta$. 
Record the displacement angle $\theta$ for both the north and south poles of the needle.
(You may need to tap the face of the compass lightly to be sure the needle hasn't become stuck.)
%\end{enumerate}

%\begin{enumerate}[labparts,resume,rightmargin=2.5in]
\item The displacement angle $\theta$ you just measured is the angle between ${\bf B}_h$ and
${\bf B}_h+{\bf B}_c$.  In the space below, sketch a picture of these vectors, and
use it to convince yourself that
\begin{equation}
%\begin{displaymath} 
B_h = \frac{B_c}{\tan\theta} 
%\end{displaymath}
\end{equation}
\answerspace{1in}
%\end{enumerate} 

%\begin{enumerate}[labparts,resume]
\item Average the north and south angles; use this average to calculate 
$B_h$ from equation (3).

\item Repeat your measurement of $\theta$ (and thus $B_h$) for currents of 0.4~A, 0.6~A, and 0.8~A.

\item Reverse the current (by switching the two leads at the galvanometer coil)
 and repeat all of your measurements of $\theta$ and calculations of $B_h$.
 
\item Determine and record the dip angle at two separate locations in the 
classroom (using the dip angle compasses). Average these measurements and use 
the result for the dip angle $\phi$.

\item Using equation (1) above, calculate the Earth's magnetic field 
\textbf{B} for each set of data. Remember that in equation (1), $\phi$
is the \textit{dip angle}, not the displacement angle of the compass.
Calculate an average value for \textbf{B} and a standard deviation.
How far off, in terms of numbers of standard deviations, is your result
from the accepted value for Richmond, $5.1 \times 10^{-5}$~T?
\end{enumerate}

%\hrulefill

\newpage

{\centering \textbf{Data Sheet}\par}

%Diameter of Coil, D (m) \> \rule{2cm}{.1pt}  

Diameter of Coil, $D$ (m)  \rule{2cm}{.1pt}  

Radius of Coil, $R$ (m) \rule{2cm}{.1pt} 

Number of turns of Coil, $N$  \rule{2cm}{.1pt}

Dip Angle, \( \phi  \) (\( ^{\circ } \)): reading 1 \rule{1cm}{.1pt}
~~reading 2 \rule{1cm}{.1pt}
~~~~Average Dip Angle, \( \phi  \) (\( ^{\circ } \)): \rule{2cm}{.1pt}

\vspace{0.3cm}
%{\centering 
\begin{tabular}{|c|c|c|c|c|c|c|}
\hline 
Current  &
Coil Field &
North Angle &
South Angle &
Average Angle &
Horizontal &
Earth's Field \\
(A)&
B\( _{c} \) (T)&
\( \theta  \)\( _{N} \) (\( ^{\circ } \))&
\( \theta  \)\( _{S} \) (\( ^{\circ } \))&
(\( ^{\circ } \))&
Component B\( _{h} \) (T)&
\textbf{B} (T)\\
\hline 
& & & & & &\\
\hline 
& & & & & &\\
\hline 
& & & & & &\\
\hline 
& & & & & &\\
\hline 
& & & & & &\\
\hline 
& & & & & &\\
\hline 
& & & & & &\\
\hline 
& & & & & &\\
\hline
\end{tabular}
%\par}
\vspace{0.3cm}

\vspace{15mm}
Earth's Magnetic Field (measured), $\left\langle {\bf B} \right\rangle$ (T) \rule{2cm}{.1pt}

Standard Deviation on Measurement, \( \sigma _{B} \) (T) \rule{2cm}{.1pt}

Number of Standard Deviations from Accepted Value, $\displaystyle \frac{\left| \left\langle B \right\rangle - B_{accepted}\right| }{\sigma _{B}}$:
\rule{2cm}{.1pt} 

\textbf{Show calculations for one trial here:}
