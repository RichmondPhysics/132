\setcounter{equation}{0}

\section{Magnetic Field of the Earth}

Name \rule{2.0in}{0.1pt}\hfill{}Section \rule{1.0in}{0.1pt}\hfill{}Date
\rule{1.0in}{0.1pt} 

\textbf{Objective}

\begin{itemize}
\item A measurement of the earth's magnetic field.
\end{itemize}
\textbf{Introduction} 

The magnetic field lines of a bar magnet, though emanating from its
full length, are densest near the poles. These lines are typically
not perpendicular to the faces of the magnet. The Earth, as you know,
is like a giant bar magnet, with its magnetic south pole near its
geographic north pole.

The field lines of the Earth's magnetic field, therefore, tend to
point obliquely at any given spot on its surface. The angle (let's
call it \( \phi  \)) that a line makes with a surface is known as
its \emph{dip angle} (see figure). Thus, each locality on earth has
its characteristic value for \( \phi  \) with regard to the terrestrial
magnetic field.

\begin{center} \begin{picture}(200,70) \put(0,0){\line(1,0){200}} \put(150,50){\vector(-2,-1){100}} \put(150,50){\vector(0,-1){50}} \put(150,50){\vector(-1,0){100}} \put(100,60){\makebox(0,0){$B_h$}} \put(160,25){\makebox(0,0){$B_v$}} \put(90,30){\makebox(0,0){$\bf B$}} \put(75,6){\makebox(0,0){$\phi$}} \put(60,0){\oval(10,10)[tr]} \end{picture} \end{center}

The horizontal component of the Earth's magnetic field,

\begin{equation}
B_h = B\cos\phi,
\end{equation}

causes the magnetized needle of a compass to align in the geographic
north-south direction.

%%This is about 1000 times more complicated than it has to be!
% by producing a torque on it. Recall that the
%magnitude of a torque is \( \tau  \) = r F\( \sin \theta  \), where
%F is the force, r is the distance from the axis of rotation to the
%point at which the force acts, and \( \theta  \) is the angle between
%the line from the axis of rotation to the point of interaction and
%the direction of the force.
%
%\begin{center} \begin{picture}(200,75) \put(0,0){\line(0,1){10}} \put(150,0){\line(0,1){10}} \put(65,5){\vector(-1,0){65}} \put(85,5){\vector(1,0){65}} \put(75,5){\makebox(0,0){$\bf r$}} \put(0,15){\circle*{3}} \put(80,55){\makebox(0,0){$\tau = r F \sin\theta$}} \put(150,15){\vector(0,1){50}} \put(150,15){\vector(3,4){37.5}} \put(160,20){\makebox(0,0){$\theta$}} \put(130,35){\makebox(0,0){$F \sin\theta$}} \put(170,50){\makebox(0,0){$\bf F$}} \thicklines \put(0,15){\line(1,0){200}} \end{picture} \end{center}
%
%The force on the compass is due to the Earth's magnetic field,
%F = pB, where p is the so-called \emph{pole strength} (recall that
%F = qE, for static electric fields, where q is the charge). In this
%case, the force acts on both ends of the magnet, so that r (the distance
%from the axis of rotation to the point at which the force acts) is
%half the length of the compass needle, $l$; \( \theta  \) is the
%angle between the alignment of the compass needle and the direction
%of B\( _{h} \). Hence, the torque on a compass needle due to the
%Earth's magnetic field is
%
%\begin{displaymath} \tau_e = plB_h\sin\theta. \end{displaymath}
%

You will use a tangent galvanometer to produce an additional horizontal
magnetic
field perpendicular to the Earth's field (in other words, one that
points east or west).
A tangent galvanometer consists of a vertical, circular coil of wire
with N turns, a pedestal compass at the center of the coil, and electrical
contacts so that a direct current can be established in the coil.
The magnetic field at the center of the galvanometer coil caused by a current
I in the coil is

\begin{equation}
B_c = N\frac{\mu_0I}{2R}
\end{equation}

where R is the radius of the coil and \( \mu _{0} \) is the permeability
of free space (1.25664 x 10\( ^{-6} \) T\( \cdot  \)m/A). B\( _{c} \)
is perpendicular to the vertical plane of the coil. If the plane of
the coil is placed in the north-south direction (that is, parallel
to the compass needle when there is no current in the coil), then,
when there is current in the coil, the compass will be influenced
by two perpendicular fields, B\( _{h} \) and B\( _{c} \). 
The compass needle will now point in the direction of the vector
sum ${\bf B}_h+{\bf B}_c$.  

Let $\theta$ be the angle through which the compass needle turns
when the current in the tangent galvanometer is turned on.
In other words, $\theta$ is the angle between ${\bf B}_h$ and
${\bf B}_h+{\bf B}_c$.  Sketch a picture of these vectors, and
use it to convince yourself that

%The
%latter provides a second torque (see figure, checking carefully the
%angles, \( \cos \theta =\sin (\frac{\pi }{2}-\theta ) \) ):
%
%\begin{displaymath} \tau_c = plB_c\cos\theta \end{displaymath}
%
%\begin{center} \begin{picture}(150,150) \put(10,75){\line(1,0){100}} \put(60,25){\line(0,1){100}} \put(5,75){\makebox(0,0){W}} \put(60,130){\makebox(0,0){N}} \put(115,75){\makebox(0,0){E}} \put(60,20){\makebox(0,0){S}} \put(34,40){\vector(3,4){54}} \put(34,40){\vector(-1,0){10}} \put(34,40){\vector(0,-1){10}} \put(88,112){\vector(1,0){10}} \put(88,112){\vector(0,1){10}} \put(65,90){\makebox(0,0){$\theta$}} \multiput(34,40)(12,-9){4}{\line(4,-3){10}} \multiput(88,112)(12,-9){4}{\line(4,-3){10}} \multiput(60,75)(12,-9){2}{\line(4,-3){10}} \put(15,40){\makebox(0,0){$\scriptstyle pB_c$}} \put(108,112){\makebox(0,0){$\scriptstyle pB_c$}} \put(34,25){\makebox(0,0){$\scriptstyle pB_h$}} \put(88,127){\makebox(0,0){$\scriptstyle pB_h$}} \put(92,81){\makebox(0,0){$\scriptstyle r$}} \put(104.5,43){\makebox(0,0){$\scriptstyle l$}} \put(89,77){\vector(-3,-4){10}} \put(95,85){\vector(3,4){10}} \put(100,37){\vector(-3,-4){20}} \put(109,49){\vector(3,4){20}} \end{picture} \end{center}
%
%When current is established in the coil, the needle will rotate through
%an angle \( \theta  \) with respect to the plane of the coil until
%the opposing torques are equal in magnitude. Thus, when equilibrium
%is established, we have
%
%\begin{displaymath} \tau_e = \tau_c \Rightarrow plB_h\sin\theta = plB_c\cos\theta \end{displaymath}

\begin{equation}
%\begin{displaymath} 
B_h = \frac{B_c}{\tan\theta} 
%\end{displaymath}
\end{equation}

\vskip 1in

By measuring the angle $\theta$, we can determine $B_h$.  
If we then measure the dip angle $\phi$, we can determine
the Earth's magnetic field from equation (1) above.

\textbf{Apparatus}

\begin{itemize}
\item tangent galvanometer 
\item power supply 
\item DC ammeter (0-1 amp)
\item banana plug leads (1 red, 2 black) with alligator clips
%\item switch
\item compass
\item ruler 
\item dip angle compass 
%\item wooden stand
\end{itemize}
\vspace{15mm}
\textbf{Activity}

\begin{enumerate}
\item Measure and record on the accompanying data sheet the diameter D of
the galvanometer coil. Calculate and record the radius R of the coil.
\item Count and record the number of turns, N, of the coil.
\item With the power supply off, connect its positive and negative terminals 
to the tangent galvanometer and  a DC ammeter (in series). 
Be sure the ammeter is connected in \underline{series} with 
the coil. Use long wires to connect the outside terminals of the coil
to the circuit so that the coil may be removed from the magnetic effects
of the ammeter and power supply. Be sure no magnetic material other
than the compass is in the vicinity of the coil. Turn the coarse current 
control on the power supply all the way up and the voltage controls all the 
way down.
\item Place the compass on the platform of the tangent galvanometer and rotate  
the galvanometer until the compass needle is aligned with the plane of the wire 
coil of the galvanometer. Rotate the compass body so that the needle points 
north and south on the compass.
\item Turn on the power supply and slowly turn up the fine voltage control 
until the current is about 0.2 A as measured with the orange ammeter.
Note that the compass needle rotates. It might be a good idea to tap
lightly the face of the compass to be sure the needle hasn't become
stuck. Record the current and the displacement angles at both the
north and south poles of the needle. Calculate B\( _{c} \) from equation (2).
\item Average the north and south angles; use this average to calculate 
B\( _{h} \) from equation (3).
\item Repeat steps 5 and 6 for currents of 0.4 A, 0.6 A, and 0.8 A.
\item Reverse the current (by switching the two leads at the galvanometer coil)
 and repeat steps 5, 6 and 7.
\item Determine and record the dip angle at two separate locations in the 
classroom (using the dip angle compasses). Average these measurements and use 
the result for the dip angle, \( \phi  \).
\item Using equation (1) above, calculate the Earth's magnetic field 
\textbf{B} for each set of data. Remember that in equation (1), \( \phi  \) 
is the \underline{dip angle}, not the displacement angle of the compass.
Calculate an average value for \textbf{B} and a standard deviation.
How far off, in terms of numbers of standard deviations, is your result
from the accepted value for Richmond, $5.1 \times 10^{-5}$ T?
\end{enumerate}
\hrulefill

\newpage

{\centering \textbf{Data Sheet}\par}

%Diameter of Coil, D (m) \> \rule{2cm}{.1pt}  

Diameter of Coil, D (m)  \rule{2cm}{.1pt}  

Radius of Coil, R (m) \rule{2cm}{.1pt} 

Number of Turns of Coil, N  \rule{2cm}{.1pt}

Dip Angle, \( \phi  \) (\( ^{\circ } \)): reading 1 \rule{1cm}{.1pt}
~~reading 2 \rule{1cm}{.1pt}

Average Dip Angle, \( \phi  \) (\( ^{\circ } \)): \rule{2cm}{.1pt}

\vspace{0.3cm}
{\centering \begin{tabular}{|c|c|c|c|c|c|c|}
\hline 
Current  &
Coil Field &
North Angle &
South Angle &
Average Angle &
Horizontal &
Earth's Field \textbf{B} \\
(A)&
B\( _{c} \) (T)&
\( \theta  \)\( _{N} \) (\( ^{\circ } \))&
\( \theta  \)\( _{S} \) (\( ^{\circ } \))&
(\( ^{\circ } \))&
Component B\( _{h} \) (T)&
(T)\\
\hline 
&
&
&
&
&
&
\\
\hline 
&
&
&
&
&
&
\\
\hline 
&
&
&
&
&
&
\\
\hline 
&
&
&
&
&
&
\\
\hline 
&
&
&
&
&
&
\\
\hline 
&
&
&
&
&
&
\\
\hline 
&
&
&
&
&
&
\\
\hline 
&
&
&
&
&
&
\\
\hline
\end{tabular}\par}
\vspace{0.3cm}

\vspace{15mm}
Earth's Magnetic Field (measured), <\textbf{B}> (T) \rule{2cm}{.1pt}

Standard Deviation on Measurement, \( \sigma _{B} \) (T) \rule{2cm}{.1pt}

Number of Standard Deviations from Accepted Value, \( \frac{\left| <B>-B_{accepted}\right| }{\sigma _{B}} \):
\rule{2cm}{.1pt} 

\textbf{Show calculations for one trial here:}
