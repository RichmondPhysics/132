
\section{Electric Charges, Fields, and Potentials of a Dipole}

Name \rule{2.0in}{0.1pt}\hfill{}Section \rule{1.0in}{0.1pt}\hfill{}Date
\rule{1.0in}{0.1pt}

\textbf{Objective}

\begin{itemize}
\item To investigate the electric field and potential of an electric dipole.
\end{itemize}

\textbf{Apparatus}

\begin{itemize}
%\item Electric field and potential simulation entitled {\it Charges and Fields}.
\item An internet browser with access to the Physics Education Technology (PhET) website
\end{itemize}

\textbf{Introduction}

In the previous unit we studied the dependence
of the electric field and the electric potential on $r$, the distance from
a charge distribution.
Now we will study the same ideas for a charge distribution commonly found in physics and chemistry
using the same methods we used before.

\textbf{Activity 1: The Electric Field of a dipole}

(a) Run the program {\it Charges and Fields} at the Physics Education Technology (PhET)
website. Go to the following address {\tt https://phet.colorado.edu/en/simulation/charges-and-fields}
and click the {\it Download} button to start the program in your browser.
A blank `table top' with a set of menu buttons will appear. 
If you don't see this consult your instructor.

(b) Go to the check boxes at upper-right on the table top and check the {\tt Electric Field}
and {\tt Grid} boxes.

(c) Construct an electric dipole by placing two charges of magnitude 1 nC and opposite sign
on the left hand side of the table top.
Click and drag the charges 
from the box at lower-middle of the table top placing them in a vertical 
line 1 meter apart with the positive charge on the top. 
Use the {\tt Tape Measure} on the right-hand side of the table top
to determine the distance between the major grid lines.
Drag it over to your charge and then 
drag one of the crosses on the {\tt Tape Measure} to the center of one charge.
Click and drag the other cross to the center of one of other one to get
the distance between the crosses.

(d) {\bf Prediction} How will the electric field be oriented between the two charges? 
How will the field be oriented outside the region of the two charges?
How will the field depend on $r_\perp$, the perpendicular distance from the 
midpoint of a line joining the two charges?
\vspace{25mm}

(e) Place a few sensors (the yellow dots in the box at lower-middle of the table top)
along a line that is \underline{perpendicular to the midpoint of a line joining the two charges}.
This is the $r_\perp$ axis.
The size and direction of the arrow on each sensor represent the magnitude and direction of the electric 
field at that point due to the dipole.
In what direction do the arrows point?
\vspace{15mm}

(f) Check the {\tt Values} box on the right-hand side of the table top.
You should see the value of the field appear near each sensor.
Click on many points along the same $r_\perp$ axis so that you get a wide range of 
magnitudes from large (barely fits on the table top) to small.

(g) Use the {\tt Tape Measure} to measure the distance from the origin of the $r_\perp$
axis to each sensor and record it along with the electric field value in the table below.

(h) Print the table top (click on the three-bar menu icon in the lower-right corner)
and attach it to this unit.

\vspace{0.3cm}
{\centering \begin{tabular}{|c|c|c|}
\hline 
~~~Distance from Charge Center (m)~~~&
~~~Measured E (N/C)~~~\\
\hline
\hline 
&
\\
\hline 
&
\\
\hline 
&
\\
\hline 
&
\\
\hline 
&
\\
\hline 
&
\\
\hline 
&
\\
\hline 
&
\\
\hline 
&
\\
\hline
\end{tabular}\par}
\vspace{0.3cm}


(h) \textbf{Prediction}: From Coulomb's Law, we expect the spatial variation
of the field strength to obey a power law: \( \left| E\right| =Ar^{n} \),
where \( A \) and \( n \) are constants. What do you predict the
value of \( n \) to be?\vspace{15mm}

(i) Graph your results. Using the power fitting
function, determine the power of the function, $n$, and record it here.
Attach the plot to this unit.
\vspace{15mm}

(j) Does your result agree with your prediction? Explain any discrepancy.\vspace{15mm}

(k) How do your results compare with the power law constant you found
for the point charge? Explain.\vspace{15mm}


\textbf{Activity 2: The Electric Potential}

(a) Click the yellow {\tt Reset} button in the lower-right corner of the table top
to erase the electric field vectors. 
Turn off the electric field vectors using the check box on the right-hand side.

(b) Place the same two charges on the table top, but now arrange them horizontally with the
positive charge on the left and the negative charge on the right.
Put the charges 1 m apart and on the left-hand side of the table top.

(c) \textbf{Prediction}: You will now take measurements of the potential.
How do you expect the electric potential to change with distance from the 
electric dipole \underline{along the axis of the dipole} (the line joining 
the two charges defines the axis)?
The is the $r_\parallel$ axis.
The origin of the $r_\parallel$ is the midpoint between the charges.
\vspace{15mm}
 
(d) Use the {\tt Voltmeter} with the cross-hairs on the right-hand side of the table top
to measure the electric potential at many points along the $r_\parallel$ axis.
At each point find the distance of the cross-hairs from the origin of the $r_\parallel$ axis using
the {\tt Tape Measure} (or use the grid).
Record your measurements in the table below.

\vspace{0.3cm}
{\centering \begin{tabular}{|c|c|c|}
\hline 
~~~Distance from Charge Center (m)~~~&
~~~Measured \( \Delta  \)V (volts)~~~\\
\hline
\hline 
&
\\
\hline 
&
\\
\hline 
&
\\
\hline 
&
\\
\hline 
&
\\
\hline 
&
\\
\hline 
&
\\
\hline 
&
\\
\hline 
&
\\
\hline
\end{tabular}\par}
\vspace{0.3cm}


(f) \textbf{Prediction}: From Coulomb's Law and the definition of the
electric potential, we expect the spatial variation of the potential
to obey a power law: \( \Delta V=Br^{m} \), where \( B \)
and \( m \) are constants. What do you predict the value of \textbf{\( m \)}
to be?\vspace{15mm}


(g) Graph your results. Using the power fitting
function, determine the power of the function, $m$, and record it here.
\vspace{15mm}

(h) Does your result agree with your prediction? Explain any discrepancy.
\vspace{15mm}

(i) How do your results compare with the power law constant you found
for the point charge? Explain.\vspace{15mm}


\textbf{Activity 3: Equipotential Lines and Field Lines}

(a) Click the yellow {\tt Reset} button in the lower-right corner of the table top
to erase the electric field vectors. 
Turn off the electric field vectors using the check box on the right-hand side.

(b) Reconstruct the same charge distribution (the electric dipole)
you made in Activity 2.
Use the {\tt Voltmeter} to measure the electric potential at the same points
you did in Activity two, but now after you position the cross-hairs click
the {\tt Pencil} icon.
A curve will be drawn representing the equipotential.
Map out the
equipotential lines by moving the {\tt Voltmeter} along the $r_\parallel$ axis  
and clicking the {\tt Pencil} at regular intervals.

(c) What do these curves represent?
\vspace{15mm}

(d) Check the {\tt Electric Field} box at upper-right.
How are the field lines and the equipotential lines related to one
another at the points where they cross?
\vspace{15mm}

(e) Use the {\tt Voltmeter} to draw some equipotentials near the positive 
charge. What is the sign of the equipotentials near the positive charge?
What is the sign of the equipotentials near the negative charge?
What is the value of the electric potential along the vertical line
halfway between the two charges?
\vspace{15mm}

(f) What topographic words might you use to describe the regions around the
positive and negative charges?
\vspace{15mm}

(g) Print the result and attach it to this unit.
