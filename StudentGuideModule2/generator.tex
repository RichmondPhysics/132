
\section{The Generator}

\makelabheader %(Space for student name, etc., defined in master.tex)

\textbf{Objectives}

\begin{itemize}
\item To understand Faraday's Law of Induction.
\item To discover how an electric generator works.
\end{itemize}
\textbf{Introduction} 

The amount of magnetic field that passes (or the number of magnetic
field lines that pass) perpendicularly through a bounded region is
known as the magnetic flux through that area. Mathematically, the
flux is given by

{\raggedright \begin{displaymath} \Phi = B_\perp A = BA\cos\theta \end{displaymath}\par}

where the symbol $\perp$ indicates the perpendicular component and
$\theta$ is the angle between the direction of the magnetic field
and the normal to the surface of the area (see figure below). 

\begin{center} \begin{picture}(150,150) \put(25,50){\line(5,1){100}} \put(25,50){\line(-2,5){15}} \put(125,70){\line(-3,4){26.5}} \put(10,87.5){\line(5,1){88.5}} \multiput(0,65)(15,3){9}{\line(5,1){10}} \multiput(85,45)(-7.5,15){5}{\line(-1,2){5}} \put(68,79){\vector(1,4){15}} \multiput(22.5,69.5)(15,3){6}{\vector(0,1){40}} \multiput(23,10)(15,3){6}{\line(0,1){40}} \put(40,120){\makebox(0,0){$\bf B$}} \put(82,145){\makebox(0,0){normal}} \put(72,110){\makebox(0,0){$\theta$}} \put(68,98){\oval(10,10)[tr]} \put(110,75){\makebox(0,0){$A$}} \end{picture} \end{center}

When the boundary of the region is a loop of wire, and the amount
of flux through it changes, an emf (or voltage) is induced
in the loop. Faraday's law states that the magnitude of the emf in
a single loop is proportional to the amount of flux change in one
unit of time: $\varepsilon = -\Delta\Phi/\Delta t$, where the minus
sign indicates that the polarity of the coil voltage opposes the flux
change (Lenz's Law). That is, if the flux decreases, the emf will
cause current to flow in the coil in such a way (using the right-hand
rule) as to create a magnetic field that compensates for the loss.
If the flux increases, the emf will cause the current to flow in the
opposite direction.

\begin{itemize}
\item Faraday's law was given for a single loop coil; what is Faraday's
law for a coil of N turns? \vspace{15mm}

\item Assume the coil is rotating clockwise around the axis OO$'$ (see figure
below). In what direction (towards O or towards O$'$) does the induced
current flow when the coil has rotated through an angle of 90$^\circ$
from the position shown? \vspace{15mm}

\end{itemize}
\begin{center} \begin{picture}(100,100) \put(30,60){\line(0,1){10}} \put(50,70){\line(0,1){10}} \put(45,35){\line(0,1){10}} \put(30,60){\line(2,1){20}} \put(30,70){\line(2,1){20}} \put(45,45){\line(2,1){20}} \put(30,60){\line(-1,2){15}} \put(30,70){\line(-1,2){15}} \put(50,80){\line(-1,2){15}} \put(45,35){\line(1,-2){15}} \put(45,45){\line(1,-2){15}} \put(65,55){\line(1,-2){15}} \put(15,42.5){\line(2,1){20}} \put(35,52.5){\line(0,1){8}} \put(35,60.5){\line(2,1){20}} \put(55,70.5){\line(0,-1){16}} \put(55,54.5){\line(-2,-1){18}} \put(37,45.5){\line(0,1){14}} \put(37,59.5){\line(2,1){20}} \put(57,69.5){\line(0,-1){11.5}} \put(57,58){\line(2,1){25}} \put(38,81){\makebox(0,0){\bf N}} \put(59,45){\makebox(0,0){\bf S}} \multiput(-5,30)(20,10){6}{\line(2,1){10}} \put(-10,25){\makebox(0,0){O}} \put(115,90){\makebox(0,0){O$'$}} \multiput(15,40)(2,1){3}{\circle*{5}} \multiput(15,37.5)(1,.5){7}{\line(2,-5){5}} \multiput(80,72.5)(2,1){3}{\circle*{5}} \multiput(80,70)(1,.5){7}{\line(2,-5){5}} \put(20,20){\makebox(0,0){{\small brush}}} \put(80,85){\makebox(0,0){{\small slip ring}}} \end{picture} \end{center}

A generator converts mechanical power into electrical power. It generally
consists of magnets, a coil (usually wound around a rotating unit called an 
armature), and a device (like a pair of slip rings and brushes) to connect 
the coil to terminals.

\begin{itemize}
\item What happens to the potential difference across the terminals as the
armature rotates? Explain. \vspace{15mm}

\item Recalling that magnetic field lines run from north pole to south pole,
what will be the magnitude of the emf as it passes through the position
shown in the drawing above. \vspace{15mm}

\item At what angle, $\theta$, will the current be a maximum?\vspace{15mm}

\end{itemize}
The change in flux through a single loop is determined by taking the
derivative of the first equation:

\begin{displaymath} \Delta\Phi = \Delta (BA\cos\theta) = -BA\Delta\theta\sin\theta. \end{displaymath}

Then the emf is $\varepsilon = BA\omega\sin\theta$, since $\varepsilon = -\Delta\Phi/\Delta t$
and $\Delta\theta/\Delta t = \omega$. So, the induced emf is proportional
to $\sin\theta$.

\begin{itemize}
\item What is the induced emf in terms of $\sin\theta$ for a coil with $N$
turns? \vspace{15mm}

\item \textbf{Prediction}: Sketch a graph of the induced emf you expect
in the multiple-loop armature of a generator as a function of angle,
$\theta$. Make your prediction for angles: $0^\circ < \theta < 360^\circ$.
Under the reasonable assumption that the resistance is constant as
the coil rotates, what do you expect a graph of the current in the
coil as a function of $\theta$ to look like?\vspace{15mm}

\end{itemize}
\textbf{Note}: The model generator you will use in this experiment
consists of a coil which rotates in a nearly uniform field produced
by two sets of permanent horseshoe magnets. A spring-and-ratchet mechanism
rotates the coil in equal steps: the spring providing a torque and
a ratchet wheel turning 10$^\circ$ each time the ratchet is released.
The spring is loaded by pushing down on a catch arm, and the ratchet
is released by tipping it to one side or the other.

\textbf{Apparatus}

\begin{itemize}
\item model generator
\item galvanometer 
\end{itemize}
\textbf{Activity}

\begin{enumerate}
\item Connect the armature terminals of the model generator to the galvanometer,
which in this configuration measures current.
\item Rotate the coil through a 10$^\circ$ interval. Observe and record
(under the Deflection (I) column of the table below) the maximum deflection
of the galvanometer. The maximum value will be reached instantaneously,
so observe carefully along with at least one lab partner. It gets
easier with practice.
\item Repeat the procedure for each 10$^\circ$ rotation of the coil over
one complete revolution. The numbers on the wheel are 1/10 the angles
of the coil with respect to the vertical. Reload the spring after
every second rotation to ensure the spring tension, and therefore,
the speed of rotation remains the same. Hold the wheel while reloading
to prevent it from moving. Note that the deflection of the galvanometer
corresponds to the position of the coil at the middle of its movement.
\item Take another set of data and record it under Deflection (II).
\item Average the two readings for each angle and record in the appropriate
column.
\item Plot Average Deflection (y-axis) \textit{vs.}~angle (x-axis) and draw a smooth
curve as best you can through the points.
%\item Also plot Average Deflection $\times \sin\theta$ vs. angle.
\item How do the curves compare to your prediction? Explain.\vspace{15mm}

\item If the resistance of the model generator coil were doubled, while
the size, shape and number of turns remained the same, what would
be the effect on the e.m.f. produced in the coil?\vspace{15mm}

\item If the coil resistance were doubled, what would be the effect on the
galvanometer deflection?\vspace{15mm}

\item What effect would changing the speed of rotation have on the galvanometer
deflection?\vspace{15mm}

\end{enumerate}
\vspace{0.3cm}
\begin{center} 
\begin{tabular}{|c|c|c|c|c|c|}
\hline 
Coil Setting&
Deflection&
Deflection&
Average &
\( \sin \theta  \)&
Avg. Deflection\\
(\( \theta  \)) {[}degrees{]}&
(I)&
(II)&
Deflection&
&
\( \times  \) \( \sin \theta  \)\\
\hline 
&
&
&
&
&
\\
\hline 
&
&
&
&
&
\\
\hline 
&
&
&
&
&
\\
\hline 
&
&
&
&
&
\\
\hline 
&
&
&
&
&
\\
\hline 
&
&
&
&
&
\\
\hline 
&
&
&
&
&
\\
\hline 
&
&
&
&
&
\\
\hline 
&
&
&
&
&
\\
\hline 
&
&
&
&
&
\\
\hline 
&
&
&
&
&
\\
\hline 
&
&
&
&
&
\\
\hline 
&
&
&
&
&
\\
\hline 
&
&
&
&
&
\\
\hline 
&
&
&
&
&
\\
\hline 
&
&
&
&
&
\\
\hline 
&
&
&
&
&
\\
\hline 
&
&
&
&
&
\\
\hline 
&
&
&
&
&
\\
\hline 
&
&
&
&
&
\\
\hline 
&
&
&
&
&
\\
\hline 
&
&
&
&
&
\\
\hline 
&
&
&
&
&
\\
\hline 
&
&
&
&
&
\\
\hline 
&
&
&
&
&
\\
\hline 
&
&
&
&
&
\\
\hline 
&
&
&
&
&
\\
\hline 
&
&
&
&
&
\\
\hline 
&
&
&
&
&
\\
\hline 
&
&
&
&
&
\\
\hline 
&
&
&
&
&
\\
\hline 
&
&
&
&
&
\\
\hline 
&
&
&
&
&
\\
\hline 
&
&
&
&
&
\\
\hline 
&
&
&
&
&
\\
\hline 
&
&
&
&
&
\\
\hline
\end{tabular}\vspace{0.3cm}

\end{center}

