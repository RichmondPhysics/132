\section{An Example Problem Using Electric Potential}

\begin{comment}
This lab is really just a worksheet I've used in my 132 class. --Matt Trawick, 6/2015
\end{comment}

\makelabheader %(Space for student name, etc., defined in master.tex)

\vspace{0.1in}
\textbf{Objective and Statement of Problem:}
 
An ionized helium atom (mass $m=6.7\times 10^{-27}$~kg, charge $q_0=+1.6\times 10^{-19}$~C) is released from rest 20~cm away from a +60~nC point charge.  What is the helium ion's speed $v$ when it is 60~cm away from the point charge?

\textbf{Solution:}

Your first instinct might be to find the force on the helium ion, and then get the acceleration and velocity from there.  But that would be hard, because the force is different everywhere, leading to an acceleration that is not constant.  It's much easier to use the idea of electric potential and conservation of energy!

\answerspace{0.1in}

\textbf{Step 1:} 

Find the electric potential $V_i$ at the place 20~cm from the point charge, and $V_f$ at 60~cm from the point charge.  (Your answers should be in volts.)
\answerspace{1.9in}


\textbf{Step 2:}

Use the idea of conservation of energy to write the final kinetic energy of the helium ion $K_f$ in terms of its charge $q_0$ and the potential difference $V_f-V_i$.  What is your numerical answer, in Joules?
\vspace{1.6in}

\textbf{Step 3:}

So what's the final speed of the helium ion, in meters per second?
\answerspace{1.4in}
