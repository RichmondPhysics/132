%% LyX 1.1 created some parts of this file.  For more info, see http://www.lyx.org/.
%% Do not edit unless you really know what you are doing.

\usepackage[T1]{fontenc}
\usepackage[nomarginpar]{geometry}
\usepackage{tocloft} %Allow us to leave page numbers for Parts out of table of contents
\cftpagenumbersoff{part} %No page numbers for Parts out of table of contents
\renewcommand{\cftsecdotsep}{\cftsubsecdotsep}
%\usepackage{newclude} %Allows use of /include*{}
%DANGER DANGER: newclude is NOT compatible with package xr, used for external references.
\geometry{verbose,letterpaper}
\usepackage{fancyhdr}
\usepackage{babel}
\setlength\parskip{\medskipamount}
\setlength\parindent{0pt}
\usepackage{graphicx}
\usepackage{wrapfig}
%\usepackage{epstopdf} %this package apparently allows pdflatex to work on this document, since all we use are eps figures.
\usepackage{comment}
\usepackage{esvect}
\usepackage{amsmath} %uncommented by MT 5/2015, used in "E near charged rod"
\usepackage{mathtools} %added by MT 6/2015, for access to dcases environment in finding_v_from_e
\usepackage{tabularx} %added by MT 6/2015, for fixed width columns, used in rc_circuits
\usepackage{microtype}
\usepackage{titlesec}
\usepackage{xr}
\usepackage{makeidx}

%For fixed width columns:
\newcolumntype{L}[1]{>{\raggedright\arraybackslash}p{#1}}
\newcolumntype{C}[1]{>{\centering\arraybackslash}p{#1}}
\newcolumntype{R}[1]{>{\raggedleft\arraybackslash}p{#1}}


\addtolength{\oddsidemargin}{1.0cm} %without these two lines, larger margin is on the OUTSIDE.  We want the larger edge on the INSIDE, to allow room for the three hole punches
\addtolength{\evensidemargin}{-1.0cm}

\setlength\topmargin{0.2in}
\addtolength{\hoffset}{-1.0cm}
\addtolength{\textwidth}{2.0cm}
\addtolength{\voffset}{-1.5cm} %This line is apparently needed on some versions of MikTex XeLatex.  Comment out if your pages appear shifted too high.
\addtolength{\textheight}{3.5cm}
% define a strut for extra vertical space in tables.
\newcommand{\hi}{\rule[-2mm]{0mm}{6mm}}

\pagestyle{fancy}
%\fancyhead[LE,RO]{\slshape \rightmark} %This is the default for fancy page style
%\fancyhead[LO,RE]{\slshape \leftmark}
\fancyhead[LO,RE]{\slshape \rightmark} 
\fancyhead[LE,RO]{\slshape \leftmark} % Reversed LE, RO to  LO,RE to make headers come out correctly on even/odd pages



%%%%%%%%%%%%%%%%%%%%%%%%%%%%%% LyX specific LaTeX commands.
\providecommand{\LyX}{L\kern-.1667em\lower.25em\hbox{Y}\kern-.125emX\@}
\newenvironment{LyXParagraphIndent}[1]%
{
  \begin{list}{}{%
    \setlength\topsep{0pt}%
    \addtolength{\leftmargin}{#1}
    \setlength\parsep{0pt plus 1pt}%
  }
  \item[]
}
{\end{list}}
%% Special footnote code from the package 'stblftnt.sty'
%% Author: Robin Fairbairns -- Last revised Dec 13 1996
\makeatletter
\let\SF@@footnote\footnote
\def\footnote{\ifx\protect\@typeset@protect
    \expandafter\SF@@footnote
  \else
    \expandafter\SF@gobble@opt
  \fi
}
\expandafter\def\csname SF@gobble@opt \endcsname{\@ifnextchar[%]
  \SF@gobble@twobracket
  \@gobble
}
\edef\SF@gobble@opt{\noexpand\protect
  \expandafter\noexpand\csname SF@gobble@opt \endcsname}
\def\SF@gobble@twobracket[#1]#2{}
\makeatother


%I make use of some latex features to manage the section numbers. To use those you have to insert the following lines into the latex preamble (before the %"\begin{document}" command).

% two new commands to do labelling. - gpg 12/4/13
\newcommand{\customlabel}[2]{%
\protected@write \@auxout {}{\string \newlabel {#1}{{#2}{}}}}

\newcommand{\actlabel}[1]{%
\protected@write \@auxout {}{\string \newlabel {#1}{{\arabic{activity}}{}}}}

\newcommand{\makelabheader}
%{Name: \rule{2.0in}{0.1pt}\hfill{}Section: \rule{1.0in}{0.1pt}\hfill{}Date: \rule{1.0in}{0.1pt}}
{Name: \rule{2.0in}{0.1pt}\hfill{}Lab Partner(s): \rule{3.0in}{0.1pt}}

%\newcommand{\dir131}{../../131/StudentGuideModule1} %This does not work, because commands can only be made of numeric characters, not numbers.

%A new command for putting a box around a paragraph:
\newenvironment{newboxed} %maybe there's a better way to do this.  I just cribbed from the web. --MT
    {\begin{center}
    \begin{tabular}{|p{0.9\textwidth}|}
    \hline\\
    }
    { 
    \\\\\hline
    \end{tabular} 
    \end{center}
    }

\newcounter{activity}

%  The following command, \answerspace, should be used to replace \vspace.
%  \vspace{} is not ideal for an answer space for students, for two reasons:
%  1. It can be ignored if it comes at the end of a page, and
%  2. The spacing is exact, and Latex will not stretch or compress it at all to make things fit on a page, which means
%  that other things WILL get stretched or compressed to make things fit, which means those other things will 
%  end up looking bad, and leading to a lot of underfull \vbox warnings.
%  \answerspace fixes both of those problems, specifically allowing the space to grow to up to 1.5 times the stated size.

\newlength\answerlength
\newcommand{\answerspace}[1]{
	\setlength\answerlength{#1}
	\vspace*{1.0\answerlength plus 0.5\answerlength}}

%  The next several lines implement \includelab, which replaces \include.
%  Usage is \includelab{1}{file} to include it, or \includelab{0}{file} to NOT include it.  
%  But all 0's can be overridden by writing \includealllabstrue in the master.tex file, which is easier than deleting 
%  fifty individual `%' signs and then remembering to put them all back, which is what you had to do before.
%  \includeonly still works as you expect it to.
\newif\ifincludealllabs
\newcommand{\includelab}[2]{
	\ifnum#1=1
		\include{#2}
	\else {
		\ifincludealllabs
		 	\include{#2}
		\fi}
	\fi
}
