
\section{Conservation of Angular Momentum}

\makelabheader %(Space for student name, etc., defined in master.tex)

\textbf{Objectives} 

To test the Law of Conservation of Angular Momentum and to explore the applicability
of angular momentum conservation among objects that experience no external torques. 

\textbf{Apparatus}

\begin{itemize}
\item A video analysis system
\end{itemize}
\textbf{Overview }

As a consequence of Newton's laws angular momentum, like linear momentum, is
conserved in isolated systems. This means that, no matter how
many internal interactions occur, the total angular momentum of any system should 
remain constant if there are no external torques. 
When one of the objects gains some angular momentum another part of the system must lose the same amount. 
If angular momentum isn't conserved, there is some outside torque acting on the system. 
By expanding the boundary of the system to include the source of that torque we 
can always preserve the Law of Angular Momentum Conservation. 

In this unit you will test the notion of the conservation of angular momentum.
As in tests of the conservation of linear momentum, we will investigate what
happens when two bodies undergo a ``rotational'' collision.
You will view a movie where a large weight is dropped onto a rotating disk and determine the moment of
inertia, the angular speed, and finally, the angular momentum of the rotator-disk-weight
system before and after this perfectly inelastic collision.

\textbf{Activity 1: Analyzing a Movie of the Collision} 

(a) Download the movie {\it spinningCollision.wmv} from the location\raggedright

{\verb!https://facultystaff.richmond.edu/~ggilfoyl/genphys/132/links.html!}

\noindent and store it on your computer {\tt Desktop}.

(b) Analyze the movie using the tools described in the Appendix. Start by scaling it using 
the ruler in the movie frame which has a length $l_{ruler} = 0.381~m$.
Be careful to place the origin of your coordinate system on the axle of the
rotator so the angular displacement you measure will be the desired one.

\textbf{Activity 2: The Moment of Inertia Before and After the Collision}

(a) Calculate the theoretical value of the moment of inertia of the metal disk
using measurements of its radius and mass. The mass of the disk is $M_d = 5.10~kg$. 
Use the video analysis tools to measure the radius $r_d$.
Be sure to state units and
show the expression you used!
\vspace{5mm}

\( r_{d} =\)  \hfill{}\( M_{d}= 5.1~kg\) \hfill{}
\vspace{5mm}

\( I_{d}= \)
\vspace{5mm}

(b) The rotating fixture that holds the disk has a complex shape. 
We have determined its moment of inertia without the disk and recorded the result here. 
\vspace{5mm}

\( I_{f} = 0.0020~kg-m^2\)
\vspace{5mm}

(c) After dropping the weight on the rotating disk, the system will have a new
moment of inertia. Derive a formula for the moment of inertia of a cylindrical-shaped 
weight of mass \( m_{w} \) and radius \( r_{w} \) revolving about the origin at a distance \( r_{r} \). 
The distance $r_r$ is from the center of the rotator to the center of the dropped weight.
(You will have to use the parallel axis theorem to do this.)
\vspace{5mm}

\( I_{w} =\)  
\vspace{5mm}



\parbox{6.5in}{(d) The mass of the weight is $m_w = 1.0 ~kg$. 
Use the movie image to measure its diameter and record it
here.

\vspace{5mm}

\( m_{w} = 1.0~kg\)  \hfill{}\(r_{w} =\) \hfill{}
\vspace{5mm}}

(e) Come up with a formula for the moment of inertia $I$ of the whole system
before and after the collision and calculate the moment of inertia before the
collision only. (The moment of inertia after the collision will be determined at the end of the experiment, 
because you will not know \( r_{r} \) until then.) 
Don't forget to include the units in \( I_{before} \).
\vspace{5mm}

\( I_{\mbox{\small before}}= \) 
\vspace{5mm}

\( I_{\mbox{\small after}} =\)  
\vspace{5mm}

(b) Check that you placed the origin of your coordinate system on the axle of the
rotator so the angular displacement you measure will be the desired one. 
Use a marker at or near the edge of the disk to record position for each frame. 
The resulting data should contain three columns with the values of time, $x$-position, and $y$-position.
Calculate the angular position for each time using the $x-$ and $y-$position data, fit the results,
and extract the angular speed of the rotator before the `collision'.

\( \omega _{\mbox{\small before}}=\)

\vspace{5mm}

(b) Now follow a similar procedure to determine the angular speed after the collision.
Fit the data and extract the angular speed of the rotator after the `collision'.

\( \omega _{\mbox{\small after}}=\)

\vspace{5mm}

(c) Determine the distance of the center of the weight from the center of the rotator \( r_{r} \). 
To do this, measure the distance from the center
of the rotator to the edge of the weight \( r_{edge} \) and use the result for
the radius of the weight \( r_{w} \). 
Calculate the distance from the origin to the center of the weight \( r_{r} \) (it is just the 
sum of \( r_{edge} \) and \( r_{w} \)). 
Use these results and those above to calculate the final moment of inertia.
\vspace{5mm}

\( r_{\mbox{\small edge}}  =\) \hfill{}\( r_{w} \) = \hfill{}\( r_{r} \) =\hfill{}
\vspace{5mm}

\( I_{\mbox{\small after}} =\)  
\vspace{5mm}

\textbf{Activity  3: Calculation of Angular Momentum}

(a) Calculate the angular momentum before and after the collision, including UNITS. 
Calculate the average of your two measurement $L_{ave}$ and use that result to
calculate the percent difference between the two results. Record it below. 
Go around to the other lab groups and get their results for the percent difference between 
the angular momenta before and after the collision.
Make a histogram of the results you collect and calculate the average and standard deviation.
For information on making histograms, see Appendix \ref{excel}. For information on calculating the average and
standard deviation, also see  Appendix \ref{treatment}. Record the average and standard deviation here.
Attach the histogram to this unit.
\vspace{5mm}

\( L_{\mbox{\small before}}= \)  \hfill \(L_{ave}=\) \hfill
\vspace{5mm}

\( L_{\mbox{\small after}}= \)
\vspace{5mm}

\( \frac{\Delta L}{L}= \frac{L_{after} - L_{before}}{L_{ave}} \)  
\vspace{10mm}

(b) What is your expectation for the difference between the initial and final angular momentum?
Do the data from the class support this expectation? 
Use the average and standard deviation for the class to quantitatively answer this question.
\vspace{20mm}


(c) Within the limits of experimental uncertainty, is angular momentum 
conserved?  Be quantitative.
\vspace{20mm}

(c) Would the procedure you followed above change if the weight was moving 
horizontally at a constant velocity when you dropped it? 
If it changed, what would be different?
\vspace{20mm}

