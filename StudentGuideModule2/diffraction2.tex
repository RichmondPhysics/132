\section{Diffraction 2}

Name \rule{2.0in}{0.1pt}\hfill{}Section \rule{1.0in}{0.1pt}\hfill{}Date
\rule{1.0in}{0.1pt}

\textbf{Objective}

\begin{itemize}
\item To investigate the diffraction of light from a single slit and to see how 
this diffraction combines with interference from multiple slits.
\end{itemize}

\textbf{Apparatus}

\begin{itemize}
\item Basic optics diode laser
\item Optical bench with rotary motion sensor
\item Phototransistor for measuring light intensity (mounted on rotary motion 
sensor
\item ``Single Slit Set'' slit accessory
\item ``Multople Slit Set'' slit accessory
\item {\it DataStudio} 750 Interface
\end{itemize}

\textbf{Introduction}

The set-up for this experiment is the same as for the Interference of Light 
experiment. Mount the laser on the optical bench on the opposite end from the 
rotary motion sensor. 

\textbf{Activity 1 - Single Slit Diffraction}

(a) On the ``Single Slit Set'' slit accessory, select a slit of width .08 mm. 
Rotate the wheel so that this single slit is at the center of the opening. 
Position the single slit about 70 cm from the phototransistor mount. Measure 
the distance from the slit to the phototransistor carefully, realizing that 
the phototransistor lies 25.4 mm behind the opening in the mount. This is the 
distance $L$ below. Record the distance here.
\vspace{10mm}

(b) Turn on the laser and adjust the beam direction so that it falls on the 
single slit you have selected. (There are two adjustment screws on the back 
of the laser.) You should see the diffraction pattern on the phototransistor 
mount.

(c) Position the phototransistor mount so the diffraction pattern is at the 
same height as the opening in the center of the phototransistor mount. The 
phototransistor is mounted behind this hole. Slide the phototransistor mount 
back and forth to make sure that it stays centered on the diffraction pattern. 
Then set it at one side of the pattern to begin the experiment. Start the 
``Interference'' activity in the \textbf{132 Workshop} folder. When you are 
ready, click \textbf{Start} and move the phototransistor from one side of the 
slide to the other as you did in the Interference of Light experiment, taking 
4-5 seconds to complete the motion. Click \textbf{Stop}. The diffraction 
pattern will appear on the computer screen. In theory, the pattern should look 
like Fig. 3 of the Diffraction of Light experiment (also see p. 921 of your 
textbook). Add the title ``Single Slit Diffraction Pattern'' and the axis 
labels ``Relative Intensity'' (vertical) and ``Position (m)'' (horizontal). 
Print the graph and attach it to this unit.

(d) Use the \textbf{Smart Tool} to measure the positions of the first minima 
on either side of the central maximum and record them here. Estimate an 
uncertainty in each measurement.
\vspace{15mm}

\newpage

(e) Determine the difference between these two measurements (with uncertainty).
\vspace{15mm}

(f) The width of the central maximum is given by (see text p. 922)

\[\Delta y = \frac{2 \lambda L}{a}\]

where \( \lambda \) in the wavelength of the laser beam, $L$ is the distance 
from the slit to the phototransistor, and $a$ is the slit width. Using your 
data from above, calculate the wavelength of the laser beam and its uncertainty.
\vspace{30mm}

\textbf{Activity 2 - Double Slit Interference/Diffraction Pattern}

(a) Replace the ``Single Slit Set'' accessory with the ``Multiple Slit Set'' 
accessory that you used in the Interference of Light experiment.  Select a 
double slit of width .04 mm and separation .125 mm. Position the double slit 
about 70 cm from the phototransistor, measure the distance carefully, and run 
the scan as you did in the Interference experiment. The result should look like 
Fig. 4 of the Diffraction of Light experiment. This is the same pattern you did 
in Activity 2 of the Interference experiment. Include an appropriate title and 
axis labels and print the graph. Use the \textbf{Smart Tool} to read the 
positions of the maxima in the left column of the table below, with the 
central peak half way down the column.

\vspace{0.3cm}
{\centering \begin{tabular}{|c|c|}
\hline
Position Reading (m)&
Change in Reading (m)\\
\hline
\hline
&
\\
\hline
&
\\
\hline
&
\\
\hline
&
\\
\hline
&
\\
\hline
&
\\
\hline
&
\\
\hline
\end{tabular}\par}
\vspace{0.3cm}

(b) Calculate the difference between each pair of adjacent readings (as you did 
in the Interference experiment) and record it in the right column of the above 
table. Calculate the average and standard deviation of the separation between 
adjacent peaks and record them here.
\vspace{20mm}

(c) Calculate the wavelength of the laser beam as you did in the Interference 
experiment. Determine an uncertainty and express both in nanometers.






