
\section{The Diffraction Grating}

\makelabheader %(Space for student name, etc., defined in master.tex)

\bigskip
\textbf{Objective}

\begin{itemize}
\vspace{-0.12in}  %There's gotta be a better way to tame this list spacing than this!  --MT
\item To determine the wavelength of laser light using a diffraction grating.
\end{itemize}

\textbf{Apparatus}
\begin{itemize}
%\setlength\itemsep{-3pt}
%\vspace{-0.12in}  %There's gotta be a better way to tame this list spacing than this!  --MT
\item Diffraction Grating 
\item Laser Pointer
\item Thermometer Clamp
\item Lens Holder
\item Lab Stand
\item 1-meter stick 
\item 2-meter stick
\end{itemize}

\textbf{Introduction}

Light bends (a bit) around corners. This phenomenon is called diffraction.
Interference, or the overlap of waves, is the basis for diffraction.
In a transmission grating, lines, about 4,000 to 8,000 per centimeter,
are ruled onto glass. The unruled portions of the glass act as slits.
The interference, and thus diffraction, which results from shining
a beam of light through the grating permit the measurement of the
wavelength of the light. The relevant relationship, known as the grating
equation is:
\begin{displaymath} n\lambda = d \sin \theta \end{displaymath}
where $n$ is the order of the spectrum (the number of bright spots
from the center), $\lambda$ is the wavelength in nanometers (10\( ^{-9} \)
meters), $d$ is the separation in nanometers between grating lines,
and $\theta$ is the angle of deviation from the light beam's original
direction through the grating (the angle of diffraction).

\bigskip
\textbf{Activity}

\begin{enumerate}
\item Record the separation between grating lines: \( d= \)
\item Turn on the laser, being careful to avoid looking directly into the
beam or shining it at anyone. Aim the light beam through the diffraction
grating so that a horizontal series of dots appears on the wall. Adjust
the positions of the laser and grating until you easily see at least
two dots on either side of the brightest (central) dot.
\item Are the dots of the interference/diffraction pattern the same intensity?
Describe the pattern you observe.\vspace{15mm}

\item Measure and record the distance from the grating to the wall, $L$, in 
the table below, as well as the distances from the central dot to the first 
dot to the right, $x$, and the first dot to the left, $x'$. Compute the average
of these $x$ values and record it as $x_{ave}$.
\item Compute and record a value for $\theta$ by using the appropriate 
trigonometric relation between $L$ and $x_{ave}$. Then, compute $\sin \theta$. 
Finally, compute the wavelength using the equation above.
\item Repeat the procedure for three additional values of $L$.
\item Compute the average of your four determinations of the laser light's
wavelength. Compute the standard deviation and consider it an uncertainty in 
your measurement. Express the result as $\lambda$ = average plus or minus 
uncertainty.

\end{enumerate}
\vspace{15mm}
\begin{center}
\begin{tabular}{|c|c|c|c|c|c|c|}
\hline 
\( L \) (cm)&
\( x \) (cm)&
\( x' \) (cm)&
\( x_{ave} \) (cm)&
\( \theta  \) (deg)&
\( \sin \theta  \)&
\( \lambda  \) (nm)\\
\hline
\hline 
&
&
&
&
&
&
\\
\hline 
&
&
&
&
&
&
\\
\hline 
&
&
&
&
&
&
\\
\hline 
&
&
&
&
&
&
\\
\hline
\end{tabular}\vspace{0.3cm}

\end{center}
