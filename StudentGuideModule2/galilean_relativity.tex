
\section{Galilean Relativity}

\makelabheader %(Space for student name, etc., defined in master.tex)

\textbf{Objective }

To investigate the relationship between different inertial reference
frames and develop the equations of Galilean relativity.

\textbf{Overview}

Before we begin our study of Einstein's special theory of relativity
we will first investigate the effects of observing a phenomenon in
two different inertial frames of reference. A spring-launched projectile
will be {}``fired'' from a moving platform and we will discover
how the description of this phenomenon changes(or doesn't change)
in a frame that moves along with the platform. You will use the video
analysis software and mathematical modeling tools to find the equations
that describe the horizontal motion (x vs. t) and the vertical motion
(y vs. t) of the projectile in a stationary frame and then in an inertial
frame moving at constant speed with the launcher. The set of relationships
between the inertial reference frames forms Galilean relativity. These
relationships provide an intuitive, {}``common sense'' picture of
the world that works well at low velocities, but fails in many surprising
ways at high velocities. At high velocities we must resort to Einstein's
special theory of relativity that we will discuss later. To do the
activities in this unit you will need:

\begin{itemize}
\item A video analysis system (\emph{VideoPoint}).
\item The film Moving Launcher.
\item Graphing and curve fitting software.
\end{itemize}
\textbf{Activity 1: Observing Projectile Motion From a Moving Launcher}

(a) Use the \emph{VideoPoint} package to analyze the film \emph{Moving
Launcher} and determine the position of the projectile in each frame
with a fixed origin. To do this task follow the instructions of \textbf{Appendix
\ref{videopoint}: Video Analysis} for calibrating the film and analyzing the data.
When you are analyzing the movie, place the fixed origin in the first
frame of the movie at the dark spot on the cart the launcher rides.
This makes the comparison with the activities below easier. Change
the origin by performing the following steps.

\begin{enumerate}
\item Click on the arrow icon near the top of the menu bar to the left.
The cursor will be shaped like an arrow when you place it on the movie
frame.
\item Click at the origin (where the axes cross) and drag the origin to
the desired location.
\item Click on the circle at the top of the menu bar to the left to return
to the standard cursor for marking points on the film.
\end{enumerate}
(b) Collect the data for analysis by following the instructions in
\textbf{Appendix \ref{videopoint}}. The data table should contain three columns with
the values of time, x-position, and y-position. Print the data table
and attach it to this unit.

(c) Determine the position of the launcher at the first and last frames
of the movie. Using these results, what is the horizontal and vertical
speed of the launcher during the movie?

\vspace{0.3cm}
{\centering \begin{tabular}{p{20mm}p{20mm}p{30mm}p{70mm}}
x\( _{0} \)= &
x\( _{1} \)=&
\( \Delta  \)t=&
v\( _{xlauncher} \) =\\
&
&
&
\\
y\( _{0} \)=&
y\( _{1} \)= &
&
v\( _{ylauncher} \) =\\
\end{tabular}\par}
\vspace{0.3cm}

(d) Plot and fit the position versus time data for the horizontal
and vertical positions of the projectile. Print each
plot and attach it to this unit. Record the equation of each fit here.
Be sure to properly label the units of each coefficient.

x(t) =
\vspace*{5mm}

y(t) =
\vspace{5mm}

What is the horizontal speed of the projectile? How did you determine
this?
\vspace{2in}

\textbf{Activity 2: Changing Reference Frames}

(a) We now want to consider how the phenomenon we just observed would
appear to an observer that was riding along on the launcher at a constant
speed. Assume the moving observer places her origin at the same place
you put your origin on the first frame of the movie. Predict how each
graph will change for the moving observer.

Horizontal position versus time:
\vspace{10mm}

Vertical position versus time:
\vspace{10mm}

(b) Use the \emph{VideoPoint} package to analyze the film \emph{Moving
Launcher} again and determine the position of the projectile in each
frame. However, this time you will use a moving origin that is placed
at the same point on the cart on each frame. Use the same point on
the launcher that you used to define the origin in the first frame
during the previous activity. To change the origin from frame to frame
follow these instructions.

\begin{enumerate}
\item Open the movie as usual and enter one object to record. First you
must select the existing origin and change it from a fixed one to
a moving one. Click on the arrow near to top of the menu bar to the
left. The cursor will have the shape of an arrow when you place it
on the movie frame. Click on the existing origin (where the axes cross)
and it will be highlighted.
\item Under the \textbf{Edit} menu drag down and highlight \textbf{Edit
Selected Series}. A dialog box will appear. Click on the box labelled
\textbf{Data Type} and change the selection to \textbf{Frame-by-Frame}.
Click OK.
\item Click on the circle at the top of the menu bar to the left to change
the cursor back to the usual one for marking points. Go to the first
frame of interest. When the cursor is placed in the movie frame it
will be labelled with {}``Point S1''. Click on the object of interest.
The film will NOT advance and the label on the cursor will now be
{}``Origin 1''. Click on the desired location of the origin in that
frame. The film will advance as usual. Repeat this procedure to accumulate
the x- and y-positions relative to the origin you've defined in each
frame.
\end{enumerate}
(c) Collect the data for analysis. The data table should contain three
columns with the values of time, x-position, and y-position.
Print the data table and attach it to this unit.

(d) Plot and fit the position versus time data for the horizontal
and vertical positions. 
When you have found a good fit to the data, record your
result below, print the graph, and attach a copy to this unit. Be
sure to properly label the units of each coefficient in your equation.

x(t) =
\vspace{5mm}

y(t) =
\vspace{5mm}

What is the horizontal speed of the projectile? How did you determine
this?
\vspace{60mm}

\textbf{Activity 3: Relating Different Reference Frames}

(a) Compare the two plots for the vertical position as a function
of time. How do they differ in appearance? Are the coefficients of
the fit for each set of data different? Do these results agree with
your predictions above? If not, record a corrected {}``prediction''
here.
\vspace{20mm}

(b) Compare the two plots for the horizontal position as a function
of time. How do they differ in appearance? Are the coefficients of
the fit for each set of data different? Do these results agree with
your predictions above? If not, record a corrected {}``prediction''
here.
\vspace{17mm}

(c) What is the difference between the horizontal velocities in the
two reference frames? How does this difference compare with the horizontal
velocity of the launcher? How are the horizontal velocities of the
projectile in each inertial reference frame and the velocity of the
launcher that you determined above related to one another? Does this
relationship make sense? Why or why not?
\vspace{15mm}

(d) Consider a point ${\bf r} = x{\bf i} + y{\bf j}$ on the
ball's trajectory in the stationary observer's reference frame. If
the moving observer's time frame is moving at the speed \( v_{x\ launcher} \)
then what would the moving observer measure for $x$? Call this horizontal
position of the moving observer $x'$.
\vspace{15mm}

(e) What would the moving observer measure for $y$? Call this vertical
position of the moving observer $y'$.
\vspace{15mm}

(f) The relationships you found above are from Galilean relativity. You
should have obtained the following results.

{\centering \( x'=x-v_{xlauncher}t \)\par}

{\centering \( y'=y \)\par}

{\centering \( v_{x}=v_{x}'+v_{xlauncher} \)\par}

The primes refer to measurements made in the moving frame of reference
in this case. If you did not get these expressions consult your instructor.

\textbf{Activity 4: Testing Galilean Relativity}

(a) You can test your mathematical relationships with the spreadsheet
capabilities of \emph{Excel}. Use your data for the stationary
observer and the relationship you derived to calculate what the observer
moving with the launcher would measure. As an example of \emph{Excel}'s
spreadsheet functions, consider graphing a function like \( f(x)=2x-1 \)
where the values of x are entered as data in column 1 in \emph{Excel}.
To calculate \( f(x) \) and place the result for each value of x
in column 2, double click on the box at the top of column 2 that has
the lone {}``2'' in it. The cursor will now appear in the box at
the top of the Data window. In that window use the following syntax
to calculate \( f(x)=2x-1 \)

{\centering =(2.0{*}C1)-1\par}

where {}``C1'' refers to the data in column 1 and the {}``{*}''
implies multiplication. Hit the {}``Enter'' key when you are finished.
You should see the proper series of numbers appear in column 2. If
you do not, consult your instructor.

(b) Once you know how to make spreadsheet calculations use the data
in \emph{Moving Launcher Data} to calculate the horizontal position
versus time for the moving observer. Make a plot of the calculated
time for the moving observer and fit the result.

For the transformed stationary observer data:
\vspace{5mm}

\begin{LyXParagraphIndent}{0.05\columnwidth}
x'(t) =
\vspace{5mm}

\end{LyXParagraphIndent}

For the moving observer data (see Activity 3.c):
\vspace{5mm}

\begin{LyXParagraphIndent}{0.05\columnwidth}
x(t) =
\vspace{5mm}

\end{LyXParagraphIndent}

(c) Your {}``transformed'' data for the stationary observer should
closely resemble the results for the moving observer. Is this what
you observe? If not, consult your instructor. Print and attach a copy
of your plot to this unit.
