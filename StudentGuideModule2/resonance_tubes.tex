
\section{Resonance in Tubes}

Name \rule{2.0in}{0.1pt}\hfill{}Section \rule{1.0in}{0.1pt}\hfill{}Date \rule{1.0in}{0.1pt}



{\noindent \bf Objectives:} \begin{list}{$\bullet$}{\itemsep0pt \parsep0pt}

\item Determine the resonant frequency for a tube open at one end.

\item Determine tube lengths at resonance for a tube of variable length.

\item Determine the velocity of sound in air in the laboratory (two ways).

\end{list}

\noindent {\bf Introduction} 

The Economy Resonance Tube is designed for the study of resonance in columns of air.  The tube set includes a movable inner tube with a closed end and an outer tube which is open at both ends.  The inner tube also includes a measuring tape to easily find the length of the air column in the outer tube.  To adjust the length of the outer tube, simply slide the inner tube until the desired length appears on the measuring tape.  Open tube experiments can also be performed with the outer tube by removing the inner tube.

\noindent In order that the tube resonate, the frequency of the vibrating air must coincide with the natural frequency of the tube (which may be its fundamental or one of its overtones). For the Economy Resonance Tube, which is closed at one end, this requirement is met if the tube length is an odd number of quarter wavelengths of the sound waves produced by the source ($L = \lambda/4, 3 \lambda/4, 5 \lambda/4$, etc., where $L$ is the length of the tube and $\lambda$ is the wave length of the sound). Note that if the length of the tube is gradually increased while the source is vibrating, the distance between successive resonance positions is $\lambda/2$. \\

\noindent {\small {\bf Note:} Due to edge effects at the open end of a tube, the effective length of the tube depends on the radius of the opening. Thus, $L_{eff} = L + 0.6r$, where $L_{eff}$ is the \textit{effective} length, $L$ is the length measured, and $r$ is the tube radius.} \\


{\noindent \bf Apparatus:} \begin{list}{$\bullet$}{\itemsep0pt \parsep0pt}

\item Economy Resonance Tube 
\item Open speaker
\item Sine wave generator
\item 2 banana plug leads
\item Sound sensor
\item Meter stick
\item Data Studio 750 Interface
\item Thermometer

\end{list}

\noindent Room Temperature ($^\circ$C) \hrulefill \ \  Tube radius (m) \hrulefill

\vspace{5mm}

{\noindent \bf Activity 1: Fixed tube length} \begin{enumerate}

\item Connect the open speaker to the sine wave generator using standard banana plug leads.

\item Adjust the length of the OUTER (blue) tube to 50 cm (check with meter stick).

\item Place the tube in front of the speaker in such a way that the tube is open at one end (the speaker can be set at an angle relative to the tube length).

\item Set the sound sensor inside the tube at the open end and connect it to the Data Studio interface.

\item To activate the sound sensor, perform the following sequence:  Start up \textit{DataStudio} by going to \textit{Start} $\rightarrow$ \textit{Programs} $\rightarrow$ \textit{Physics Applications} $\rightarrow$ \textit{DataStudio}.
Click on \textit{Create Experiment}, then \textit{Setup}, then \textit{Add Sensor or Instrument}. Scroll down to \textit{Sound level sensor} and select, then click \textit{OK}. Double click \textit{Graph} at left. Click \textit{Start} to begin taking data.

\item The resonant frequencies for a tube open at one end are given by $f=nv/4L$ where $n$ is an odd integer, $v$ is the velocity of sound and $L$ is the effective tube length. Start at a frequency of 130 Hz and increase until you find three resonances (for $n$ = 1, 3, 5). These will be indicated by peaks on the sound level graph. You need to carefully determine the frequency associated with each peak.  Record the results here:\vspace{10mm}

\item From your data, plot a graph of $f$ vs $n$ (using $Excel$). Fit with a linear trendline. Print the graph (with title and axis labels) and include with this unit. Use the LINEST function in $Excel$ to determine the slope of the graph and its uncertainty. From these values, calculate the velocity of sound in air, and its uncertainty, and record them here. Don't forget to use $L_{eff}$ - see Introduction.
\vspace{12mm}

\end{enumerate}

%\noindent {\Large{\bf DATA}} \\

%\noindent Room Temperature ($^\circ$C) \hrulefill \ \  Tube radius (m) \hrulefill

%\begin{center} \begin{tabular}{||c|c|c|c|c|c|c|c|c|c|c|c|c|c|c||} \hline \hline Tuning Fork & \multicolumn{4}{|c|}{First Position of} & \multicolumn{4}{|c|}{Second Position} & \multicolumn{4}{|c|}{Third Position of} & Wave- & Velocity of \\ Frequency, & \multicolumn{4}{|c|}{Resonance, m} & \multicolumn{4}{|c|}{of Resonance, m} & \multicolumn{4}{|c|}{Resonance, m} & length, & Sound in \\ \cline{2-13} Hz & 1 & 2 & 3 & Ave. & 1 & 2 & 3 & Ave. & 1 & 2 & 3 & Ave. & m & air, m/s \\ \hline \hline &&&&&&&&&&&&&& \\ \hline &&&&&&&&&&&&&& \\ \hline \hline \end{tabular} \end{center}


{\noindent \bf Activity 2: Fixed frequency} \begin{enumerate}

\item Adjust the outer tube length to 20 cm.

\item Set the speaker inside the open end of the tube so that it is closed at both ends. (Do not include sound sensor for this activity. You will just be listening for loudness.)

\item Set the sine wave generator frequency to 600 Hz (with low amplitude).

\item Slowly move the inner tube to increase the effective length of the tube. Listen for maximum loudness and record the length of the tube when resonance is achieved:\vspace{10mm}

\item Increase the length of the tube until two more resonance lengths are found for the constant frequency and record them here:\vspace{10mm}

%\item Average your two values to determine your experimental velocity of sound in m/s: \ \ \  \rule{2cm}{1pt}

\item The resonant frequencies for a tube closed at both ends are given by 
$f=nv/2L$ where $n$ = 1, 2, 3, etc. Solve this equation for $L$ and plot $L$ 
vs $n$ (using $Excel$) where $n$ = 1, 2 and 3 for the three resonance lengths 
you found in 4 and 5 above. Fit with a linear trendline. Use the LINEST function in $Excel$ to determine the slope of the graph and its uncertainty. Use these values to determine the velocity of sound in air and its uncertainty and record them here:

\vspace{20mm}

\item The velocity of sound in air at $0^\circ$C is 331.4 m/s.  The temperature dependence of sound velocity in air is given by $v(T) = 331.4 + 0.6T$, where $T$ is in $^\circ$C and $v$ is in m/s. Calculate an ``accepted'' value of the velocity of sound in air from this formula.

\vspace{20mm}

\item Does the ``accepted'' value fall within the range of uncertainty of your experimental value?

\end{enumerate}
