
\section{Introduction to LabVIEW}

LabVIEW is an extremely powerful software system for building high-performance
instrumentation and analysis applications. It integrates data acquisition,
analysis, presentation, computation, simulation, and modeling into
one versatile package. LabVIEW applications are called virtual instruments
(VIs) because they have the look and feel of actual instruments. The
user interface is provided by the front panel which specifies the
inputs and outputs with objects such as knobs, dials, switches, digital
displays, graphs, and meters. The computers are furnished with data
acquisition boards which equip LabVIEW for data acquisition and instrumentation.
Transducers are used to convert physical quantities such as distance,
force, temperature, and time into electrical signals that are read
by the data acquisition boards. In this appendix, we will briefly
describe how to run LabVIEW applications and how to use some of the
features.

\textbf{Running a VI} 

To run a VI: 1. Open the VI by double-clicking on the icon, or by
launching LabVIEW, selecting Open from the File menu and then selecting
the VI. 2. Click the single arrow pointing to the right in the upper
left-hand corner of the front panel. This arrow is called the Run
button. The arrow will become filled while the VI is executing and
will return to its original state when the VI is finished.

\textbf{Changing the Value of a Digital Display }

There are two ways to enter or change the value of a digital display.
1. You can click inside the digital display window with the pointer
and then enter numbers from the keyboard. The enter button appears
to remind you that the new value replaces the old value only when
you press the enter key on the keyboard or click outside the display
window. 2. You also can click on the increment buttons (little up
and down arrows on the left side of the display window) with the pointer
to change the least significant digit of the display.

\textbf{Using Cursors} 

The cursor feature of LabVIEW graphs is very useful for determining
numerical values from the graphs. To display the cursors, place the
pointer in the graph window, press the apple key, and hold the mouse
button down. This is called command-clicking . A pop-up menu will
appear. Select Show and a submenu will appear. Select Display Cursors
from this submenu. The cursor panel will appear. If the cursors do
not appear in the graph window, click on the box next to the lock
icon on the cursor panel. When you click on this box and hold the
mouse button down, a submenu appears which can be used to change the
cursor attributes or to center the cursors in the graph window. Each
vertical cursor is teamed with a horizontal cursor, and a cursor marker
appears where they intersect. The cursors can be dragged with the
mouse to different positions on the graph and the coordinates of the
cursors can be read from the cursor panel. You can hide the cursors
with the same procedure used to show them.

\textbf{Scaling Graphs} 

The graph scales may be setup for automatic scaling. When the AutoScale
option is set, the scale limits and number of markers are computed
with each graphing operation. When AutoScale is turned off, the scale
remains the same with each graphing operation. To change the AutoScale
option, command-click in the graph window and select the desired setting
in the X Scale or Y Scale submenu.

You can also change the limits for a particular graph manually by
sweeping across the limit value with the mouse button depressed (the
value will be highlighted) and typing the new value on the key board
followed by the enter key. Limits set in this way will not be retained
with each graphing operation.

\textbf{Quitting LabVIEW} 

To exit LabVIEW, select Quit from the File menu and don t save anything
if you are asked.
