
\section{Electrostatics}

Name \rule{2.0in}{0.1pt}\hfill{}Section \rule{1.0in}{0.1pt}\hfill{}Date
\rule{1.0in}{0.1pt}

\textbf{Objective}

\begin{itemize}
\item To understand the basic phenomena of electric charges at rest.
\end{itemize}
\textbf{Introduction}

Atoms consist of a central nucleus made up of protons and neutrons
surrounded by one or more electrons. While the nuclei of solids are
essentially localized, some of the electrons may be free to move about.
A substance which has as many electrons as it has protons is said
to be electrically neutral. Dissimilar objects have different affinities
for electrons. When two such objects, initially neutral, are rubbed
together, the friction may cause electrons to pass from one to the
other. After separation, neither object is neutral. Each is said to
have been {}``charged by friction''. An isolated, electrified object
becomes neutral again if its electron-proton balance is restored.
A convenient means for accomplishing this is to connect the object
to earth by means of a conductor, through which electrons readily
travel. This process is called {}``grounding the body''. Since an
electrified object is referred to as {}``charged'', grounding is
also referred to as {}``discharging''.

Substances through which electrons do not move easily are called {}``non-conductors'',
or {}``insulators''. Experiment has shown that when rubber and wool
are rubbed together, electrons pass form the wool to the rubber. The
electrons remain on the surface of the rubber--a non-conductor--where
they were transferred.

Rubbing a metal rod with a wool cloth can also transfer electrons.
This rod, however, is a conductor and electrons pass through it to
the experimenter and then to the earth. People, made mostly of salt
water, are good conductors, as well. Metal that is isolated, however,
can be electrified. This can be demonstrated with an electroscope,
which has a metal knob connected to a stem from which a thin metal
leaf hangs. An insulator prevents contact of these metal parts with
the case, and consequently the earth.

\textbf{Apparatus}

\begin{itemize}
\item electroscope
\item rubber and glass rods
\item wool and silk cloth
\end{itemize}
\textbf{Activity 1: Charging by Friction}

\begin{enumerate}
\item Be sure the electroscope is discharged by touching the knob with your
finger. Explain what happened and why you are convinced the electroscope
is discharged.\vspace{15mm}

\item \textbf{Prediction:} If you rub the knob of an electroscope with a
wool cloth, what will be the state of the electroscope when you remove
the cloth? Explain.\vspace{15mm}

\item Gently and repeatedly rub the knob of the electroscope for a couple
of minutes. Remove the cloth. Note any differences in the electroscope
from its appearance before you rubbed.\vspace{15mm}

\item Explain what, if anything, happened.\vspace{15mm}

\end{enumerate}
\textbf{Activity 2: Charging by Contact}

\begin{enumerate}
\item Discharge the electroscope as before.
\item Charge the rubber rod by friction with
the wool cloth.
\item Does anything occur in the electroscope when you bring the disc close
to the knob without touching it?\vspace{15mm}

\item \textbf{Prediction:} What will happen to the electroscope if you repeatedly
touch its knob with a freshly charged object?\vspace{15mm}

\item Touch the disc to the knob; rub the disc again and again touch it
to the knob; repeat this procedure two or three more times. Describe
any changes to the electroscope.\vspace{15mm}

\item Repeat the procedure above until the electroscope's leaf is at approximately
a twenty degree angle with the stem.
\end{enumerate}
\textbf{Activity 3: Kinds of Electrification}

\begin{enumerate}
\item Electrify one end of the rubber rod by wrapping the wool cloth around
the rod, squeezing the wool against the rod, twisting the rod vigorously
to ensure good contact, and separating the wool from the rod.
\item \textbf{Prediction:} What will happen when you bring the electrified
end of the rubber rod toward, but not touching, the electroscope's
knob? What will happen if you do the same with the wool cloth?\vspace{15mm}

\item Bring the charged end of the rubber rod toward the knob, but do not
touch it. Record what happens.\vspace{15mm}

\item Repeat number 3 with the wool cloth.\vspace{15mm}

\item What differences were there between the trial with the rod and the
trial with the cloth?\vspace{15mm}

\item How would you account for these differences?\vspace{15mm}

\item \textbf{Note:} By definition, the electrical state of the rubber after
being rubbed by the wool is negative. That is, an object that has
an excess of electrons is said to be negatively charged. Realize that
this is only a convention.
\item If the rubber is said to be negatively charged, in what electrical
state is the wool cloth?\vspace{15mm}

\item How can an electroscope be used to determine the nature of any charge?\vspace{15mm}

\item Rub the end of the glass rod with the silk cloth and determine the
charge of each after they are separated.
\item What is the charge of an electron? Of a proton?\vspace{15mm}

\end{enumerate}
\textbf{Activity 4: Action of the Electroscope}

\begin{enumerate}
\item \textbf{Discussion:} Two facts explain the rise or fall of the leaves
of an electroscope: (a) Like charges repel (unlike charges attract);
and (b) Free electrons move about in a conductor when an electric
force acts upon them.
\item Discharge the electroscope.
\item \textbf{Prediction:} What will happen when you bring the charged rubber
rod near the discharged electroscope? What will happen if you do the
same with the wool cloth?\vspace{15mm}

\item Test your predictions; record the results; try to explain them.\vspace{15mm}

\item When the wool approaches the knob, which way do the free electrons
in the metal of the electroscope move (up toward the knob or down
toward the leaf)?\vspace{15mm}

\item Do the leaf and stem now become positive or negative?\vspace{15mm}

\item In Activity 3, the electroscope was negatively charged before either
the rod or the wool was brought toward the knob. For the case of the
rod, in which direction do the free electrons in the electroscope
move? Does the electron displacement increase or decrease the electrostatic
force separating the leaf from the stem?\vspace{15mm}

\end{enumerate}
\textbf{Activity 5: Charging by Induction}

\begin{enumerate}
\item Discharge the electroscope.
\item \textbf{Prediction:} What will be the effect on the electroscope if
you perform the following experiment: while grounding the electroscope
with your finger, bring an electrified rubber rod near the knob, then
take away your finger and then the rod (in that order)?\vspace{15mm}

\item Carry out the experiment and describe the result.\vspace{15mm}

\item Explain the result and why your prediction agreed or disagreed with
it.\vspace{15mm}

\item \textbf{Prediction:} Note that no electrons moved between the rod
and the electroscope. What charge has been induced on the electroscope?\vspace{15mm}

\item Test your prediction with the negatively charged rubber rod and the
positively charged wool.
\item Does the test verify or contradict your prediction?\vspace{15mm}
\end{enumerate}

