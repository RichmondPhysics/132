
\section{Refraction at Spherical Surfaces: Thin Lenses}

Name \rule{2.0in}{0.1pt}\hfill{}Section \rule{1.0in}{0.1pt}\hfill{}Date
\rule{1.0in}{0.1pt}

\textbf{Objective}

\begin{itemize}
\item To investigate thin lenses.
\end{itemize}
\textbf{Introduction} 

A lens converges or diverges light rays. It is a transparent material
bounded, in the case of thin lenses, by spherical edges. The line
between the centers of curvature of these edges is referred to as
the principal axis. The principal focus is the point on the principal
axis where parallel incident rays converge. The distance from the
lens to this point is known as the focal length. The relation between
the focal distance, $f$, the object distance, $p$, and the image distance,
$q$, is:

\begin{displaymath} \frac{1}{p} + \frac{1}{q} = \frac{1}{f}. \end{displaymath}

\textbf{Apparatus}

\begin{itemize}
\item light fence 
\item converging and diverging lenses (1 each) 
\item optical bench 
\item light bulb 
\item white file card
\end{itemize}
\textbf{Investigation 1: The Converging Lens}

\textbf{Activity 1}

\begin{itemize}
\item Arrange the light source apparatus so that the parallel rays of light
cross a piece of paper. 
\item Place a convex lens on the paper perpendicular to the central ray.
Outline its position and the path of the rays. Pay particular attention
to the condition near the principal focus.
\item What is the focal length of this lens?\vspace{15mm}

\end{itemize}
\textbf{Activity 2 }

\begin{itemize}
\item A light bulb--in particular, the lettering on a light bulb--at one
end of the optical bench will be the object in this investigation.
Choose a particularly clear letter or number on the bulb. Measure
and record its height, $h_0$.\vspace{10mm}

\item Place a converging lens in its holder, turn on the light bulb, and
position it and the lens so that a sharp image of the lettering on
the bulb appears on the card.
\end{itemize}
\vspace{0.3cm}
{\centering \begin{tabular}{|c|c|c|c|c|c|}
\hline 
~~~~~~~\( p \)~~~~~~~&
~~~~~~~\( q \)~~~~~~~&
~~~~~~~\( h_{i} \)~~~~~~~&
~~~~~~~\( \frac{h_{i}}{h_{0}} \)~~~~~~~&
~~~~~~~\( \frac{q}{p} \)~~~~~~~&
~~~~~~~\( f \)~~~~~~~\\
\hline
\hline 
&
&
&
&
&
\\
\hline 
&
&
&
&
&
\\
\hline 
&
&
&
&
&
\\
\hline 
&
&
&
&
&
\\
\hline 
&
&
&
&
&
\\
\hline
\end{tabular}\par}
\vspace{0.3cm}

\begin{itemize}
\item Make and record five measurements of image distance, $q$, for five
different values of the object distance, $p$: two with $p > q$, two
with $p < q$, and one with $p \approx q$. For each observation, measure
the height of the image of the same letter or number you measured
earlier, $h_i$. Calculate and record the ratios of the image and object
heights, $h_i / h_0$, and the image and object distances, $q / p$.
\item Calculate and record the focal length, $f$, for each observation and
determine an average focal length, $f_{ave}$.
\item What is the relationship between the ratio of the image to object
heights and the ratio of image and object distances? The first ratio
is called the magnification.\vspace{30mm}

\item Replace the converging lens with a diverging one. Try to obtain a
real image on the card.
\item Why can you not form a real image with a diverging lens?\vspace{15mm}

\end{itemize}
\textbf{Investigation 2: Lenses in Combination}

\textbf{Activity}

\begin{itemize}
\item Place a converging lens and a diverging lens together into the lens
holder. Check to see that you can get a real image with this combination.
\end{itemize}
\vspace{0.3cm}
{\centering \begin{tabular}{|c|c|c|c|c|c|}
\hline 
~~~~~~~\( p \)~~~~~~~&
~~~~~~~\( q \)~~~~~~~&
~~~~~~~\( h_{i} \)~~~~~~~&
~~~~~~~\( \frac{h_{i}}{h_{0}} \)~~~~~~~&
~~~~~~~\( \frac{q}{p} \)~~~~~~~&
~~~~~~~\( f \)~~~~~~~\\
\hline
\hline 
&
&
&
&
&
\\
\hline 
&
&
&
&
&
\\
\hline 
&
&
&
&
&
\\
\hline 
&
&
&
&
&
\\
\hline 
&
&
&
&
&
\\
\hline
\end{tabular}\par}
\vspace{0.3cm}

\begin{itemize}
\item Repeat the five sets of observations of Investigation 1, Activity
2, to get an equivalent focal length $f_{ave}^{eq}$.\vspace{40mm}

\end{itemize}
\textbf{Investigation 3: The Diverging Lens}

\textbf{Activity 1}

\begin{itemize}
\item Using the relation:
\end{itemize}
\begin{displaymath} \frac{1}{f^{eq}} = \frac{1}{f_1} + \frac{1}{f_2}, \end{displaymath}

\begin{quote}
determine the focal length of the diverging lens, $f_2$. Use $f^{eq}_{ave}$
for $f^{eq}$ and $f_{ave}$ for $f_1$.\vspace{2in}

\end{quote}
\textbf{Activity 2}

\begin{itemize}
\item Repeat the procedure of Investigation 1, Activity 1, with a concave
lens. Locate the principal focus by extending the refracted rays backwards.\vspace{30mm}

\item What is the focal length of this lens?\vspace{15mm}
\end{itemize}

