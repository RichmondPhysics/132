

\section{The Optical Spectrum of Hydrogen}

Name \rule{2.0in}{0.1pt}\hfill{}Section \rule{1.0in}{0.1pt}\hfill{}Date
\rule{1.0in}{0.1pt}


\textbf{Objective}

\begin{itemize}

\item To determine the wavelengths of the visible lines in the hydrogen spectrum using         
a spectrometer and a diffraction grating.

\item To determine the value of Rydberg's constant.

\item To compare the predicted energy levels with the measured ones.

\end{itemize}

\textbf{Introduction}

\bigskip
 
The spectral lines of the hydrogen spectrum that fall in the visible region are designated as 
the $H_\alpha$, 
$H_\beta$, $H_\gamma$, and $H_\delta$  lines.  
All (there happen to be four of them) belong to the Balmer series.  
In general, the spectrum of hydrogen can be represented by Rydberg's formula:

\begin{equation}
{1 \over \lambda} = R_H \left ( {1 \over n_f^2} - {1 \over n_i^2} \right )
\end{equation}

\noindent where $n_f$ can be any positive integer and $n_i$ takes on the values of $n_f + 1$, $n_f + 2$, 
$nf + 3$, and so on and $R_H$ is the Rydberg constant for hydrogen and equals $1.097 \times 10^7 m^{-1}$.

If one writes equation 1 twice--once, say for the $H_\alpha$ wavelength $\lambda_\alpha$, 
and once for the $H_\beta$ wavelength, $\lambda_\beta$, then one can eliminate $R_H$:
\begin{equation}
{1 \over {n_\beta^2}} = {1 \over n_f^2} - \left ( {\lambda_\alpha \over \lambda_\beta} \right )
   \left ( {1 \over n_f^2} - {1 \over n_\alpha^2} \right )
\end{equation}
Thus, once one finds $\lambda_\alpha$ and $\lambda_\beta$ so through trial and error one can
determine the value of the three $n$'s in equation 2 (recall they all must be integers and 
($n_f < n_\alpha < n_\beta$).

\vspace{0.2in}

\textbf{Activity 1: Measuring Spectral Lines}

\bigskip
 
Use the spectrometer to measure the angle (once on each side) 
for each line and get an average angle for each line.  
Calculate the wavelength of each line using the relation:
\begin{equation}
\lambda = d \sin \theta
\end{equation}
where $d$ is the diffraction grating spacing.


\vspace{0.25in}

\noindent Diffraction grating spacing $d ~ = ~\qquad\qquad\qquad${\AA}

\vspace{0.5in}

\begin{center}
\begin{tabular}{|c|c|c|c|c|c|}\hline
Line        & $\theta_{left}$     & $\theta_{right}$     & $\theta_{average}$ & Wavelength  & $n$ \\ 
            & (degrees, minutes)  &  (degrees, minutes)  & (decimal degrees)  & ({\AA}) &  \\ \hline
$H_\alpha$  &                             &                              &                    &                    &   \\ \hline
$H_\beta$   &                             &                              &                    &                    &   \\ \hline
$H_\gamma$  &                             &                              &                    &                    &   \\ \hline
\end{tabular}
\end{center}

\vspace{0.5in}

\textbf{Activity 2: Calculating the Rydberg Constant}

\bigskip
 
Using pairs of measured wavelengths and guesses for $n_f$ and one of the $n_i$'s, 
calculate the other $n_i$ in equation 2.  
When this calculated number is close to an integer you may have the correct value for the other $n$'s.
Once you have determined the proper $n$'s, calculate a value of $R_H$ for each line and compare the 
average of these with the accepted value.  
Use your results to predict the value of the next line in the series $H_\delta$.  
It's measured value is $4101.2${\AA}.  
How does your prediction compare?     

\noindent $R_\alpha ~ = ~\qquad\qquad\qquad\qquad\qquad$

\vspace{0.4in}

\noindent $R_\beta ~ = ~\qquad\qquad\qquad\qquad\qquad$

\vspace{0.4in}

\noindent $R_\gamma ~ =~ \qquad\qquad\qquad\qquad\qquad$

\vspace{0.4in}

\noindent $R_{average} ~ =~ \qquad\qquad\qquad\qquad$  \% difference = 

\vspace{0.4in}

\noindent $H_\delta ~ = ~\qquad\qquad\qquad\qquad\qquad$  \% difference =

\vspace{0.2in}


\textbf{Activity 3: The Hydrogen Level Diagram}

\bigskip
 
Make an energy level diagram showing the transitions you believe you have measured. 
How do your measured transition energies compare with the predicted ones?

