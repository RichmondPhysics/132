\section{RC Circuits}
\begin{comment}
This is an old version of an RC circuits lab.  This version is superceded by the new lab, written by Matt Trawick in 2015.  
\end{comment}

\makelabheader %(Space for student name, etc., defined in master.tex)

\textbf{Objective}

\begin{itemize}
\item To investigate the behavior of a series RC circuit, i.e. one containing a resistor and a capacitor in series with a DC power supply of fixed emf $V_{0}$.
\end{itemize}
\textbf{Introduction} 

For the charging circuit, the charge on the capacitor as a function of time
is given by
\begin{equation}
Q(t)=CV_{0}\left(1-e^{-t/RC}\right)
\end{equation}
while for the discharging circuit (no emf) the charge is
\begin{equation}
Q(t)=Q_{0}e^{-t/RC}
\end{equation}
where $Q_{0}$ is the initial charge on the capacitor.  Since $Q=CV$ for the
capacitor, the voltage across the capacitor has the same time dependence,
namely
\begin{equation}
V(t)=V_{0}\left(1-e^{-t/RC}\right)
\end{equation}
for the charging circuit and
\begin{equation}
V(t)=V_{0}e^{-t/RC}
\end{equation}
for the discharging circuit.
\vspace{5mm}

We will use the discharging circuit in this experiment.
For this circuit, the time required for $Q$ (or $V$) to decrease to
$1/e$ of its initial value (the time constant) is
\begin{equation}
\tau=RC
\end{equation}
The time required for $Q$ (or $V$) to decrease to 1/2 of its initial value (the
half-life) is
\begin{equation}
t_{1/2}=(ln2)\tau =.693\tau
\end{equation}
This will be useful because in practice it is easier to measure the half-life
than the time constant.


\textbf{Apparatus}

\begin{itemize}
\item DC power supply (low voltage)
\item Electrolytic capacitor (1000-5000 $\mu$f)
\item Voltmeter (0-3V)
\item Stop watch
\item Connecting leads
\item Voltage sensor
\end{itemize}
\textbf{Activity 1: Proof of Equation (6)}

Prove equation (6) above, beginning with equation (2).
\vspace{3in}

\textbf{Activity 2: Measurement of Half-life}

The capacitor will be charged briefly, then allowed to discharge through the
voltmeter which will act as the resistor in this case. The internal resistance
of the voltmeter (0-3V) is about 3000 ohms.  Assume this is known to within
10\% and the capacitance as marked is known to within 10\%.

\begin{enumerate}
\item Connect the voltmeter directly across the capacitor.  Electrolytic
capacitors are polarity sensitive, so be sure the terminal marked with a minus
sign is connected to the \underline{negative} terminal of the voltmeter.

\item Set the power supply to about 2.5 volts and connect its negative terminal
to the negative terminal of the capacitor.

\item Connect the positive terminal of the power supply to the corresponding
terminal of the capacitor briefly to charge it, then disconnect the positive 
lead and observe the exponential decay of the voltage.  Use a stop watch to 
measure the half-life (5 trials) as follows: switch the stop watch on when the 
voltmeter reads 2.0 volts, and switch it off when the meter reads 1.0 volt. 
Record your five measurements here.\vspace{30mm}

\end{enumerate}

\textbf{Activity 3: Determination of Time Constant}

\begin{enumerate}
\item Experimental:  Find the average of your five half-life measurements and 
its uncertainty (the standard deviation) and determine the time constant from 
equation (6) above, and its uncertainty. (Note: Since the time constant and the 
half-life are directly proportional, the fractional uncertainty in the time 
constant will be the same as the fractional uncertainty in the half-life.)
\vspace{40mm}
\item Theoretical:  Calculate the time constant and its uncertainty from 
equation (5). Do your two results agree within experimental uncertainties?
\end{enumerate}

\newpage

\textbf{Activity 4: Determination of Time Constant Using Pasco Interface and Computer}

We will now look at a graph of the discharge curve using the Pasco interface 
and computer. Plug the voltage sensor into the input channel $A$ of the Pasco 
interface. Then connect the red and black leads from the voltage sensor to the 
terminals of the capacitor (the voltmeter should still be connected to the 
capacitor as before), being sure to connect the red lead to positive and black 
lead to negative. Connect the negative lead from the power supply to the 
negative terminal of the capacitor.
\vspace{10mm}

To record voltage versus time data, go to $Start$ $\rightarrow$ $Programs$ 
$\rightarrow$ $Physics Applications$ $\rightarrow$ $Data Studio$. Select the 
option $Create Experiment$. You will see an image of the 750 interface box. 
On this image, single click on the channel $A$ input. You will see a list of 
sensors. Scroll down to the $voltage sensor$ and select by clicking $OK$. 
Finally, open a graph display by double clicking on the $Graph$ option (on 
the left).
\vspace{10mm}

Make sure your power supply is still set at around 2.5 volts. Charge the 
capacitor (by touching the positive lead from the power supply to the positive 
capacitor terminal) and click the $Start$ button to begin taking data. 
Disconnect the positive power supply lead and you will see the 
discharge curve on the screen as the capacitor discharges. When you have an 
acceptable curve, click $Stop$. Print the graph and attach it to this unit.
\vspace{10mm}

Use the $Smart$ tool to determine the time when the voltage is 2.00 volts 
and again when the voltage is 1.00 volt. The difference is the half-life. 
Record it here, with an \underline{estimated} uncertainty:
\vspace{40mm}

Now calculate the time constant and its uncertainty from equation (6) and 
compare with your value of $RC$ determined from equation (5). Do they agree 
within experimental uncertainty (i.e. do the ranges overlap)?

