
\section{Magnetism: Electric Currents and Moving Charges}

\makelabheader %(Space for student name, etc., defined in master.tex)

\textbf{Objectives}

To investigate:

\begin{itemize}
\item The effect of magnetic fields on moving charges. 
\item The effect of moving charges (currents) on magnets. 
\end{itemize}

\textbf{Apparatus} 

\begin{itemize} 
\item Bar magnet
\item Oscilloscope
\item Tangent galvanometer
\item Compass
\item Power supply
\item Banana plug leads (2) with alligator clips
\end{itemize}

%\textbf{Introduction} 

%Where did this intro come from??? It has nothing to do with the lab!
%A charged object moving through a magnetic field experiences a force
%which is proportional to the magnitude of its charge and to its speed
%perpendicular to the field: $F = qvB_\perp$. Changing the number of
%magnetic field lines--the flux--through a coil of wire results in
%a current in the wire. The direction of this current is such that
%the magnetic field it produces opposes the change in the external
%field. Similarly, varying the current in one coil (the primary) produces
%a current in another nearby coil (the secondary). The current in the
%second coil, too, will flow in a direction that creates a magnetic
%field opposing that which is changing in the first coil. These relationships
%between changing fields and currents are known collectively as electromagnetic
%induction.



\textbf{Activity 1: Magnetic Forces on Moving Charges }
\begin{enumerate}[labparts]
\item An oscilloscope is built around the principle of the cathode ray tube. It emits electrons from its back end. These are accelerated by a series of electrodes and focused to strike a fluorescent screen at its front. The result is a visible spot (or trace) indicating voltage as a function of time.

\item \textbf{Predictions}: What, if anything, will happen to the spot on
the screen if the north pole of a magnet is brought near the left
side of the oscilloscope? What will happen if you do the same with
the south pole? What about when each of the poles are brought near
to the top? {[}Please do not touch the oscilloscope with the magnet.{]} 
\vspace{30mm}

\item Turn on the oscilloscope by pressing the power button. Turn the TIME/DIV 
knob completely counterclockwise. Adjust the INTEN (intensity) and FOCUS knobs 
so that a small bright spot is formed on the oscilloscope screen by the beam of 
electrons traveling toward the screen. Do not make the spot very bright. 
Adjust the ILLUM (illumination) knob so that the grid on the screen can be seen 
clearly. Use the horizontal and vertical POSITION controls to center the spot 
on the screen.

\item Bring the N-pole of a horizontal bar magnet near, but not touching,
the left side of the oscilloscope case at the height of the spot.
Record the direction of any deflection. Repeat with the S-pole.\vspace{20mm}

\item Bring the N-pole of a vertical bar magnet near, but not touching,
the top of the oscilloscope case just above the spot. Record the direction
of any deflection. Repeat with the S-pole.\vspace{25mm}

\item Turn off the oscilloscope.
\item Did the directions of deflections meet your expectations? Explain. 
\vspace{25mm}

\end{enumerate}

\textbf{Activity 2: The Effect of Moving Charges (Currents) on Magnets}

In this investigation we will use a device known as a tangent galvanometer to make a qualitative study of the effect of current (moving charges) in a coil of wire on a compass. A sketch of the galvanometer is shown below.

\begin{center}
\includegraphics{magnetism_currents/tangent_galvanometer_bw.eps}
\par
Figure 1. Tangent galvanometer and compass.
\end{center}

\begin{enumerate}[labparts]
\item With the power supply off, connect the positive and negative terminals 
on the power supply to the two side screws on the tangent galvanometer. Turn 
the coarse current control on the power supply all the way up and the voltage 
controls all the way down.

\item Before turning the power supply on, place the compass on the platform 
in the center of the tangent galvanometer. Rotate the galvanometer until the 
compass needle is aligned with the plane of the wire coil of the galvanometer.
Turn the power supply on and slowly turn up the fine voltage control.
What do you observe? Make a sketch to show the orientation of the compass 
needle and galvanometer coil with the voltage on and off. 
(It may be easiest to do this with TOP views.)
\vspace{20mm}

\newpage

\item Turn the voltage on the power supply to zero.
Rotate the entire setup (galvanometer, compass, wires) $180^\circ$
so the contacts are on the opposite side from where they were before.
Make sure the compass needle is again aligned with the plane of the wire coil.
Slowly turn the voltage back up. What do you observe?
Make another sketch to show the orientation of the compass needle and
galvanometer coil with the voltage on and off.
\vspace{30mm}

\item The deflection of the compass when current flows in the tangent 
galvanometer implies the current creates a magnetic field.
From your observations can you tell the direction
of the magnetic field?  Explain.
\vspace{30mm}

\item Reverse the wires on the contacts of the tangent galvanometer to reverse 
the direction of the current in the coil of the galvanometer.
We will now repeat the observations from above.
With the voltage off, align the compass needle and the plane of the wire coil 
of the galvanometer as in part (b) above.
Slowly turn up the voltage.
What do you observe?
Make a sketch to show the orientation of the compass and
galvanometer coil with the voltage on and off.
\vspace{30mm}

\item Rotate the entire setup (galvanometer, compass, wires) $180^\circ$
so the contacts are on the opposite side from where they were before.
Make sure the compass needle is again aligned with the plane of the wire coil.
Slowly turn the voltage back up. What do you observe?
 Make another sketch.
\vspace{30mm}

\item Based on your observations, what happens to the magnetic field of the 
tangent galvanometer when you reverse the direction of the current?

\end{enumerate}
