%This document is a standalone worksheet used by Ted Bunn.  This file is the basis for the 
%slightly edited version, electrostatics_post2016.tex
\documentclass{article}
\parindent 0pt
\parskip 6pt
\textwidth 6.5in
\textheight 9in
\topmargin -0.5in
\oddsidemargin 0in
\evensidemargin 0in


\begin{document}


\section*{Lab 0: Electric charge}

Here are three probably-familiar facts about electric charges:

\begin{itemize}
\item There are two types of electric charge, called \textit{positive}
and \textit{negative}
\item Charges of opposite type attract each other, and charges of the same type
repel each other.
\item Some materials such as metals are \textit{conductors}. Electric charge 
is free to move around through a conductor. 
\end{itemize}

Of course, we'll treat these facts in precise mathematical detail soon
enough, but for today we're going to explore some of their consequences 
qualitatively.

The main piece of apparatus for this lab is an \textit{electroscope}. 
Here's all you need to know about the electroscope for now. It has
a metal ball at the top, which is attached to a vertical metal bar.
The metal bar has a piece of thin metal foil attached to it. 

Rub the rubber rod vigorously with the wool cloth.
This has the effect of transferring some electric charge between the
cloth and the rod, so that one of them gets positive charge and one gets
negative charge.

Rub the rubber rod on the ball of the electroscope, and then take it away.
What happens?

\vskip 1in

Using the above facts about electric charges, explain why this happens.

\vskip 1in

This is important, so if you're not sure your explanation is
right, talk to me.\footnote{Needless to say, everything we do
is important! And you should always talk to me if you're not sure your
lab answers are right. But that applies especially here, because nothing
after this will make sense if this doesn't.}

Touch the ball of the electroscope. What happens? Explain why this
happens.  In the course of your explanation, you may want to say
whether you are a conductor or not.

\vskip 1.5in

Charge up the rubber rod again. This time, bring it close to the
ball of the electroscope, without touching it, then take it away.
What happens?

\vskip 1in

Explain why this happens.

\vskip 1in


Discharge the electroscope by touching it.

Suppose that you were to try the following (don't do it yet!):

\begin{enumerate}
\item Charge up the rubber rod by rubbing it.
\item Touch it to the ball of the electroscope.
\item Charge up the rubber rod some more, by rubbing it again.
\item Bring it close to the ball of the electroscope, without
touching it.
\end{enumerate}
What do you think will happen when you do the last step? (Will the
foil move up, down, or neither?) Make a prediction before trying it,
and explain briefly why you think this.

\vskip 2in

Now try it. Did it behave as expected?

\vskip 1in

Touch the electroscope with your finger to discharge it. Now rub the glass
rod vigorously with the silk cloth, and try \textit{all} of the 
above with the glass rod. In what ways, if any, is the behavior different
from what you found with the rubber rod?

\vskip 2in

From the experiments you've done, can you tell whether the charge on
the rubber rod is the same type as the charge on the glass rod, or
the opposite type?

\vskip 1in

If you answered ``no'' to the previous question, then devise an experiment
that would answer it, using the equipment you have. Describe 
the experiment briefly.

\vskip 2in

Perform the experiment. Are the charges the same or opposite?

\vskip 1in

From the data you have, is there any way to tell whether the charge
on the rubber rod is positive or negative? What about the glass rod?

\vskip 1in

The last thing we're going to examine is \textit{charging by induction}.
Discharge the electroscope, and imagine the following experiment (but don't do it yet):
While touching the electroscope
with your finger, you bring a charged rubber rod near the ball, then
take away your finger and then the rod (in that order).

Predict the result of this experiment, and briefly explain your reasoning.

\vskip 2in

Try the experiment. Did it confirm your prediction?

\vskip 1in

Based on your reasoning, what type of charge is induced on the electroscope
at the end of this experiment (the same type as the rubber rod,
or the opposite type)?

\vskip 1in

Test this prediction. Were you right?



\end{document}
