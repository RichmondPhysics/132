\section{Expansion of a gas at constant pressure}
\begin{comment}
This lab was written by Ted Bunn for spring of 2016.  It was edited slightly by Matt Trawick for this manual in April 2016.

\end{comment}

\makelabheader %(Space for student name, etc., defined in master.tex)

\bigskip

\textbf{Introduction}

This is another of those simulation-based labs using the ``Atoms in
Motion'' program.  You will work out theoretically how much energy it
should take to heat up a gas from an initial temperature $T_i$
to a final temperature $T_f$, assuming constant pressure.  Then you will
test your predictions on the computer.

\bigskip

\textbf{Activity 1: Predictions}

(a) Suppose that we have a sample of $n$ moles of monatomic gas.
We put this sample of gas in a container with a movable
piston, and we keep the system at a constant pressure $P$.  The initial
temperature of the gas is $T_i$.  Express the initial volume $V_i$ of 
the gas in terms of $P,n,T_i$ (and constants).  

\answerspace{0.8in}

(b) Now suppose we raise the temperature of the gas to some final value
$T_f$.  What is the final volume of the gas (in terms of $P,n,T_f$)?

\answerspace{0.8in}

(c) Using the relation between work, pressure, and volume, find
the relationship between the work done on the gas and the change in temperature
$\Delta T$.

\answerspace{1in}%1+?

(d) How much internal energy did the gas have initially (when it
was at temperature $T_i$)?  How much did it have at the end (when
it was at temperature $T_f$)?  Express the change in internal energy
in terms of $\Delta T$.

\answerspace{1.1in}%was 1.5

(e) How much heat $Q$ had to be added to the gas in order to cause
this rise in temperature?  

\answerspace{1.1in}

\pagebreak[3]
(f) We define the ``molar specific heat at constant pressure,''
$C_P$, to be the amount of heat added to the gas per mole and per
degree of temperature rise.  What is the molar specific heat at
constant pressure of a monatomic gas?  (Your answer should be a
constant, not depending on $P$, $T$, or any of those things.)

\answerspace{1.4in}

(g) Is the molar specific heat at constant pressure more than, less, than, or
the same as the molar specific heat at constant volume?  Explain
briefly why this makes sense.

\answerspace{1.4in}

\textbf{Activity 2: Tests}

(a) Start up the ``Atoms in Motion'' program.  We're going to begin
by setting up a gas with given values of pressure and temperature.
Click on ``Atom''
and let there be 50 atoms of type A and none of any other type.
Click on ``Box'' and turn on both the piston and the ``floor conducts
heat'' options.  Let the piston pressure be $30\times 10^5$ Pa and
the temperature be 200 K.  Start the simulation running, and let it
run for a few minutes, until the gas seems to have settled down
to equilibrium.  (The piston will still jiggle up and down some.)

(b) At this point, the gas should be at a temperature of about 200 K.
Click ``Box'' again, and turn off the ``floor conducts heat'' option.
Now no energy can enter or leave the system.  Its temperature should remain
approximately constant at 200 K.  Turn on the $\sum$ (average) option,
and look at the numbers.  The temperature should settle down to about 200 K
and the pressure to about $30\times 10^5$ Pa,
although as before there'll still be some jiggling around these values.  
(If they don't seem to have settled down to the right values, check with me.)
Stop the simulation, and 
record the pressure, volume, and temperature of the gas here:

\answerspace{1in}

(c) Calculate the internal energy of the gas from the given values.  
(Suggestion: how many moles are there?  How is internal energy related
to the number of moles?)

\answerspace{1.5in}

\pagebreak[3]
(d) Once you have calculated the internal energy, click on the ``KE''
box.  This will tell you the total kinetic energy of all the molecules
in the gas.  What is the total kinetic energy?  
Does it agree reasonably well with what you found in the last question?
\label{ke}

\answerspace{1in}

(e) Now suppose we suddenly add an amount $Q$ of heat to the gas that's
equal to the internal energy right at this moment.  Using the rule you
found before for the molar specific heat at constant pressure, calculate
$\Delta T$, the expected rise in temperature.  What would you expect the
final temperature of the gas to be after adding this much heat?

\answerspace{1.5in}

(f) We can test this by simply raising the kinetic energy of the 
molecules.  After clicking on the ``KE'' box, you'll see that you can
add any given amount of energy to the system.  Add an amount that's
equal to the amount that's already there.  (This is the
same as suddenly adding heat to the system.) Then start the simulation going
again.  After it seems to have settled down to equilibrium, what is the
temperature?  Does this agree with your prediction?

\answerspace{1.5in}

(g) Check the total kinetic energy of the molecules (by clicking on ``KE''
again).  In the previous step, we took the initial energy (as found
in step \ref{ke}) and doubled it, so we might expect the total energy
now to be twice what it was in step \ref{ke}.  Is this the case?  Why not?
Where did the ``missing energy'' go?

\answerspace{1in}

(h) Record the final volume of the gas here.  Using the result, check
whether your answer to the previous question is correct.  (Sorry
this question's kind of vague, but if I make it more specific I'll
give away the answer to the previous question.  If you're not sure
what I'm getting at, ask me.)
