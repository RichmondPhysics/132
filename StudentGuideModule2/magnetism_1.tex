
\section{Magnetism: Qualitative Interactions and Compasses}

\makelabheader %(Space for student name, etc., defined in master.tex)

\textbf{Objectives}

\begin{itemize}
\item To investigate the characteristics of magnets.
\item To understand how a compass works.
\end{itemize}
\textbf{Introduction} 

The electric interaction, you probably know, is not the only one in
which opposites attract and likes repel. Magnetic interactions have
similar characteristics. All simple magnets, regardless of size, are
bipolar: there are two magnetic poles. Consider this question, then:
Can we talk about like and unlike as we do for electricity?

\textbf{Apparatus}

\begin{itemize}
\item 2 bar magnets 
\item 2 cylindrical magnets 
\item rods and clamps
\item wool cloth
\item rubber rod
\item string
\end{itemize}
\textbf{Activity 1: The Characteristics of Magnets}

\begin{enumerate}
\item Feel the attraction between two magnets when pulled apart after having
come together without effort on your part. Describe qualitatively
in terms of strength and separation.\vspace{15mm}

\item Feel the repulsion when one of them is turned around and pushed toward
the other. Describe as in step 1.\vspace{15mm}

\item Note and describe the difference in (strength and direction of) interactions
between the ends and the middle.\vspace{15mm}

\end{enumerate}
\textbf{Activity 2: How a Compass Works}

\begin{enumerate}
\item Identify geographic north and south.
\item Hang one of the cylindrical magnets horizontally from a horizontal rod.
\item When it comes to rest, along which geographical line does the magnet
lie? \vspace{15mm}

\item Which end (colored or uncolored) is the \char`\"{}north-seeking\char`\"{}
end?\vspace{15mm}

\item Remove the cylindrical magnet and repeat step 2 with the second cylindrical magnet. Answer, again, the questions above.\vspace{15mm}

\item What happens when you bring the \char`\"{}north-seeking\char`\"{}
end of the first magnet near the hanging one's north-seeking end?\vspace{15mm}

\item What happens when you bring the first magnet's opposite end near the
second's north-seeking end?\vspace{15mm}

\item What about the first magnet's north-seeking end near the opposite
end of the hanging one?\vspace{15mm}

\item What happens when you bring the opposite ends near one another?\vspace{15mm}

\item Define in your own words like and unlike poles?\vspace{15mm}

\item What always happens between like poles?\vspace{15mm}

\item What always happens between unlike poles?\vspace{15mm}

\item Determine with a labelled bar magnet which end of your hanging magnet
should be identified as the north pole and which the south.
\vspace{10mm}
\item Why do we identify one end of a magnet as the north pole and the other
as the south?\vspace{15mm}

\item In your own words, explain a compass.\vspace{15mm}

\item In terms of magnetism, what is the earth?\vspace{15mm}

\item Charge a rubber rod with the wool cloth and bring it near the ends
of the suspended magnet; describe its effect on the magnet.\vspace{15mm}

\item Does a south magnetic pole repel a negative electric charge?\vspace{15mm}

\item Does a north magnetic pole attract a negative electric charge?\vspace{15mm}
\end{enumerate}

