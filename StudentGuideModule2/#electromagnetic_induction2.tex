
\section{Electromagnetic Induction}

Name \rule{2.0in}{0.1pt}\hfill{}Section \rule{1.0in}{0.1pt}\hfill{}Date
\rule{1.0in}{0.1pt}

\textbf{Objectives}

To investigate:

\begin{itemize}
\item The effect of changing magnetic fields on charge and current.
\end{itemize}
\textbf{Introduction} 

A charged object moving through a magnetic field experiences a force
which is proportional to the magnitude of its charge and to its speed
perpendicular to the field: $F = qvB_\perp$. Changing the number of
magnetic field lines--the flux--through a coil of wire results in
a current in the wire. The direction of this current is such that
the magnetic field it produces opposes the change in the external
field. Similarly, varying the current in one coil (the primary) produces
a current in another nearby coil (the secondary). The current in the
second coil, too, will flow in a direction that creates a magnetic
field which opposes the change in the field of the first coil. These relationships between changing fields and currents are known collectively as electromagnetic induction.

\textbf{Apparatus} 

\begin{itemize}
\item one small wire coil
\item bar magnet
\item Pasco 750 Interface
\end{itemize}
\textbf{Activity 1: A Moving Magnet and a Coil}

\begin{enumerate}
\item Turn on the computer and launch {\it EM Induction} in the 132 Workshop in the {\bf Start} menu.

\item Place a bar magnet vertically along the axis of the small coil with
the N-pole touching the coil.

\item Start recording data and lift the bar magnet quickly straight up.

\item At the end of the data taking interval, the computer should display
a value for the electromotive force (emf) induced in the small coil.
Several trials may be required to get the correct timing between starting
the data acquisition and removing the magnet. Note and record the sign of the induced
emf.\vspace{10mm}

\item \textbf{Prediction}: If you lower the magnet, N-pole down, quickly
toward the coil, what will be the sign of the emf? \vspace{15mm}

\item Carry out the experiment, starting the data acquisition, then lowering the magnet.
Record the sign of the induced emf.\vspace{10mm}

\item Did your result confirm or refute your prediction?\vspace{15mm}

\item \textbf{Prediction}: What will happen to the emf if you perform the
same pair of experiments with the S-pole toward the coil? \vspace{15mm}

\item Perform the two experiments, lifting and lowering the magnet, with
the S-pole down. Record the sign of the induced emf in each case.\vspace{10mm}

\item How did the results compare with your predictions?\vspace{15mm}

\end{enumerate}

