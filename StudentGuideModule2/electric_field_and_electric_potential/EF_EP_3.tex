
\section{The Electric Field and the Electric Potential III}

\makelabheader %(Space for student name, etc., defined in master.tex)

\textbf{Objective}

\begin{itemize}
\item To investigate the electric field and potential of an electric dipole.
\end{itemize}

\textbf{Apparatus}

\begin{itemize}
\item Electric field and potential simulation entitled {\it EMField}.
\end{itemize}

\textbf{Introduction}

In the previous units (which we will refer to as Investigation 1 and 2) we studied the dependence
of the electric field and the electric potential on $r$, the distance from
a charge distribution.
Now we will study the same ideas for a charge distribution commonly found in physics and chemistry
using the same methods we used before.

\textbf{Investigation 3: An Electric Dipole}

\textbf{Activity 1: The Electric Field}

(a) Start the program {\it EMField} from the {\it Physics Applications} menu or use the options under the 
\textbf{Display} menu to clear the table top and delete any charges.
Go to the \textbf{Sources} menu and select \textbf{3D Point Charges}.
A blank `table top' with a set of menu 
buttons at the top and bottom will appear.

(b) Go to the {\bf Display} menu and set {\it EMField} to
{\bf Show Grid} and {\bf Constrain to Grid} if they are not already set.
These choices will make the following investigation a bit easier to perform.

(c) Clear the table top and build an electric dipole by placing two magnitude
{}``8'' charges of opposite sign a distance 4 cm apart near the lower left 
of the table top, oriented along a horizontal line. Put the positive charge 
on the left.


(d) {\bf Prediction} How will the electric field be oriented between the two charges? How will the field be oriented outside the region of the two charges?
How will the field depend on $r$, the distance from the midpoint of a line joining the two charges, at large $r$?
\vspace{25mm}

(e) Click along a line \underline{perpendicular to the midpoint of a line joining the two charges}. The size and direction of the arrow represent the magnitude and direction of the electric field at that point due to the dipole.
In what direction does the arrow point?
\vspace{15mm}

(f) Click on many points along the same line so that you get a wide range of 
magnitudes from large (barely fits on the table top) to small (barely bigger 
than a dot).

(g) Print the table top and use a ruler to measure the lengths of each of the 
arrows on your plot and the distance from the midpoint of a line joining the 
two charges. Enter these data in the following table. Use the scale at the 
bottom of the table top to convert the length of each arrow into an electric 
field magnitude. The units of the scale electric field vector are $1.0 ~ N/C$.

\vspace{0.3cm}
{\centering \begin{tabular}{|c|c|c|}
\hline 
~~~Distance from Charge Center (cm)~~~&
~~~Arrow Length (cm)~~~&
~~~Measured E (N/C)~~~\\
\hline
\hline 
&
&
\\
\hline 
&
&
\\
\hline 
&
&
\\
\hline 
&
&
\\
\hline 
&
&
\\
\hline 
&
&
\\
\hline 
&
&
\\
\hline 
&
&
\\
\hline 
&
&
\\
\hline
\end{tabular}\par}
\vspace{0.3cm}


(h) \textbf{Prediction}: From Coulomb's Law, we expect the spatial variation
of the field strength to obey a power law: \( \left| E\right| =Ar^{n} \),
where \( A \) and \( n \) are constants. What do you predict the
value of \( n \) to be?\vspace{15mm}

(i) Graph your results. Using the power fitting
function, determine the power of the function, $n$, and record it here.
Attach the plot to this unit.
\vspace{15mm}

(j) Does your result agree with your prediction? Explain any discrepancy.\vspace{15mm}

(k) How do your results compare with the power law constants you found
in Investigations 1 and 2? Explain.\vspace{15mm}


\textbf{Activity 2: The Electric Potential}

(a) Under the {\bf Display} menu click on {\bf Clean up Screen} to erase the
electric field vectors.

(b) Reverse the two charges, i.e. put the positive charge on the right, keeping 
the same distance between the charges.

(c) \textbf{Prediction}: You will now take measurements of the potential.
How do you expect the electric potential to change with distance from the 
electric dipole \underline{along the axis of the dipole} (the line joining 
the two charges defines the axis)?
\vspace{15mm}
 
(d) Click on the \textbf{Potential} option under the \textbf{Field and Potential} menu. Click on the table top and a marker will be
placed at that point and labeled with the value of the potential there.
Click on many spots on the table top along the axis of the dipole (outside the dipole itself). When you are finished print the table top.
\vspace{15mm}

(e) Measure and record in the following table the values of the distance from 
the midpoint of a line joining the two charges and the potential.

\vspace{0.3cm}
{\centering \begin{tabular}{|c|c|c|}
\hline 
~~~Distance from Charge Center (cm)~~~&
~~~Measured \( \Delta  \)V (volts)~~~\\
\hline
\hline 
&
\\
\hline 
&
\\
\hline 
&
\\
\hline 
&
\\
\hline 
&
\\
\hline 
&
\\
\hline 
&
\\
\hline 
&
\\
\hline 
&
\\
\hline
\end{tabular}\par}
\vspace{0.3cm}


(f) \textbf{Prediction}: From Coulomb's Law and the definition of the
electric potential, we expect the spatial variation of the potential
to obey a power law: \( \Delta V=Br^{m} \), where \( B \)
and \( m \) are constants. What do you predict the value of \textbf{\( m \)}
to be?\vspace{15mm}


(g) Graph your results. Using the power fitting
function, determine the power of the function, $m$, and record it here.
\vspace{15mm}

(h) Does your result agree with your prediction? Explain any discrepancy.
\vspace{15mm}

(i) How do your results compare with the power law constants you found
in Investigations 1 and 2? Explain.\vspace{15mm}


\textbf{Activity 3: Equipotential Lines and Field Lines}

(a) Under the \textbf{Display} menu click on \textbf{Clean up Screen} to erase the potential values.

(b) Under the \textbf{Field and Potential} menu, drag down to \textbf{Equipotentials}.
Click the mouse on the table top and a
line will be drawn representing the equipotential line with a label
representing the value of the electric potential. {[}If
the curve does not close (i.e., the last point drawn doesn't match
up with the starting point), consult the instructor.{]} Map out the
equipotential lines by moving the cursor across the table top away
from each charge and clicking the mouse at regular intervals.

(c) What do these curves represent?\vspace{15mm}

(d) Under the \textbf{Field and Potential} menu, click on \textbf{Field Lines}.
The field lines of the charge distribution will be drawn. {[}If \textit{EMField} takes a long time to draw one of the field lines, consult your instructor.{]}
Print the result and attach it to this unit.
 
(e) How are the field lines and the equipotential lines related to one
another at the points where they cross?


