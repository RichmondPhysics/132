
\section{The Electric Field and the Electric Potential IV}

Name \rule{2.0in}{0.1pt}\hfill{}Section \rule{1.0in}{0.1pt}\hfill{}Date
\rule{1.0in}{0.1pt}

\textbf{Objective}

\begin{itemize}
\item To investigate the electric fields of atoms.
\end{itemize}

\textbf{Apparatus}

\begin{itemize}
\item Electric field and potential simulation entitled {\it EMField}.
\end{itemize}

\textbf{Investigation 4: An {}``Atom''}

\textbf{Activity 1: The Electric Field}

(a) Start the program {\it EMField} from the {\it Physics Applications} menu or use the options under the 
\textbf{Display} menu to clear the table top and delete any charges.
Go to the \textbf{Sources} menu and select \textbf{3D Point Charges}.
A blank `table top' with a set of menu 
buttons at the top and bottom will appear.

(b) Go to the {\bf Display} menu and set {\it EMField} to
{\bf Show Grid} and {\bf Constrain to Grid} if they are not already set.
These choices will make the following investigation a bit easier to perform.

(c) Build an {}``atom'' by symmetrically surrounding
a {}``+36'' charge with four (4) {}``-9'' charges.
Make the positive ``nucleus'' by clicking and dragging a ``+9'' charge
to the same spot four times.


(d) {\bf Prediction} How will the electric field be oriented inside the atom?
How will the field be oriented outside the atom?
How will the field depend on $r$, the distance from the nucleus of the atom
at large $r$?
\vspace{25mm}

(e) Click on the \textbf{Field} option under the \textbf{Field and Potential} menu.
Next, click along a line to the right of the positive charge.
What direction does the arrow point?
\vspace{15mm}

(f) Click on many points on the table top so that you get a wide range of magnitudes from large
(barely fits on the table top) to small (barely bigger than a dot).

(g) Print the table top and use a ruler to measure the lengths of each of the arrows
on your plot and the distance from the positive nucleus. 
Enter this data in the table below.
Use the scale at the bottom of the table top to convert the length of each arrow into 
an electric field magnitude.
The units of the scale electric field vector are $1.0 ~ N/C$.

\vspace{0.3cm}
{\centering \begin{tabular}{|c|c|c|}
\hline 
~~~Distance from Charge Center (cm)~~~&
~~~Arrow Length (cm)~~~&
~~~Measured E (N/C)~~~\\
\hline
\hline 
&
&
\\
\hline 
&
&
\\
\hline 
&
&
\\
\hline 
&
&
\\
\hline 
&
&
\\
\hline 
&
&
\\
\hline 
&
&
\\
\hline 
&
&
\\
\hline 
&
&
\\
\hline
\end{tabular}\par}
\vspace{0.3cm}


(h) \textbf{Prediction}: From Coulomb's Law, we expect the spatial variation
of the field strength to obey a power law: \( \left| E\right| =Ar^{n} \),
where \( A \) and \( n \) are constants. What do you predict the
value of \( n \) to be?\vspace{15mm}

(i) Graph your results. Using the power fitting
function, determine the power of the function, $n$, and record it here.
Attach the plot to this unit.
\vspace{15mm}

(j) Does your result agree with your prediction? Explain any discrepancy.\vspace{15mm}

(k) How do your results compare with the power law constants you found
in Investigation 3 for the electric dipole? Explain.\vspace{15mm}

(l) How do your results compare with the power law constants you found
in Investigations 1-2 for the positive charge distributions? Explain.\vspace{15mm}


\textbf{Activity 2: The Electric Potential}

(a) Under the {\bf Display} menu click on {\bf Clean up Screen} to erase the
electric field vectors.

(b) \textbf{Prediction}: You will now take measurements of the potential.
How do you expect the electric potential to change with distance from the positive nucleus?
\vspace{15mm}
 
(c) Click on the \textbf{Potential} option under the \textbf{Field and Potential}
menu. Click on the table top and a marker will be
placed at that point and labeled with the value of the potential there.
Click on many spots on the table top along the line to the right of the
positive charge.
When you are finished print the table top.
\vspace{15mm}

(d) Measure and record in the table the values of the distance from the
positive charge and the potential.

\vspace{0.3cm}
{\centering \begin{tabular}{|c|c|c|}
\hline 
~~~Distance (cm)~~~&
~~~Measured \( \Delta  \)V (volts)~~~\\
\hline
\hline 
&
\\
\hline 
&
\\
\hline 
&
\\
\hline 
&
\\
\hline 
&
\\
\hline 
&
\\
\hline 
&
\\
\hline 
&
\\
\hline 
&
\\
\hline
\end{tabular}\par}
\vspace{0.3cm}


(e) \textbf{Prediction}: From Coulomb's Law and the definition of the
electric potential, we expect the spatial variation of the potential
to obey a power law: \( \Delta V=Br^{m} \), where \( B \)
and \( m \) are constants. What do you predict the value of \textbf{\( m \)}
to be?\vspace{15mm}


(f) Graph your results. Using the power fitting
function, determine the power of the function, $m$, and record it here.
\vspace{15mm}

(g) Does your result agree with your prediction? Explain any discrepancy.\vspace{15mm}


(h) How do your results compare with the power law constants you found
in Investigation 3 for the electric dipole? Explain.\vspace{15mm}

(i) How do your results compare with the power law constants you found
in Investigations 1-2 for the positive charge distributions? Explain.\vspace{15mm}

(j) Atoms, and therefore molecules, are composed of equal numbers of positive
and negative charges (protons and electrons), each with its own electric
field. In many circumstances, we can treat the whole atom as a neutral
particle with no electric field (recall, for example, our treatment
of ideal gas particles). How do the electric fields of different charges
{}``cancel'' out to reduce the magnitude of the electric field?\vspace{15mm}

