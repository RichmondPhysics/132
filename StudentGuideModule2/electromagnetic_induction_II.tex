
\section{Electromagnetic Induction II}

Name \rule{2.0in}{0.1pt}\hfill{}Section \rule{1.0in}{0.1pt}\hfill{}Date
\rule{1.0in}{0.1pt}

\textbf{Objectives}

To investigate:

\begin{itemize}
\item The effect of magnetic fields on moving charges. 
\item The effect of changing magnetic fields on charge.
\item The effect of changing currents on magnetic fields.
\end{itemize}
\textbf{Introduction} 

A charged object moving through a magnetic field experiences a force
which is proportional to the magnitude of its charge and to its speed
perpendicular to the field: $F = qvB_\perp$. Changing the number of
magnetic field lines--the flux--through a coil of wire results in
a current in the wire. The direction of this current is such that
the magnetic field it produces opposes the change in the external
field. Similarly, varying the current in one coil (the primary) produces
a current in another nearby coil (the secondary). The current in the
second coil, too, will flow in a direction that creates a magnetic
field which opposes the change in the field of the first coil. These relationships between changing fields and currents are known collectively as electromagnetic induction.

\textbf{Apparatus} 

\begin{itemize}
\item signal generator 
\item two coils, one large, one small 
\item bar magnet
\item oscilloscope
\item Pasco 750 Interface
\end{itemize}
\textbf{Activity 1: Magnetic Forces on Moving Charges }

\begin{enumerate}
\item Turn on the oscilloscope by pressing the power button. Turn the TIME/DIV
knob completely counterclockwise. Adjust the INTEN (intensity) and
FOCUS knobs so that a small bright spot is formed on the oscilloscope
screen by the beam of electrons traveling toward you. Adjust the ILLUM
(illumination) knob so that the grid on the screen can be seen clearly.
Use the horizontal and vertical POSITION controls to center the spot
on the screen.
\item \textbf{Note}: An oscilloscope is built around the principle of the
cathode ray tube. It emits electrons from its back end. These are
accelerated by a series of electrodes and focussed to strike a fluorescent
screen at its front. The result is a visible trace identifying voltage
as a function of time.
\item \textbf{Predictions}: What, if anything, will happen to the spot on
the screen if the north pole of a magnet is brought near the left
side of the oscilloscope? What will happen if you do the same with
the south pole? What about when each of the poles are brought near
to the top? {[}Please do not touch the oscilloscope with the magnet.{]} \vspace{30mm}

\item Bring the N-pole of a horizontal bar magnet near, but not touching,
the left side of the oscilloscope case at the height of the spot.
Record the direction of any deflection. Repeat with the S-pole.\vspace{20mm}

\item Bring the N-pole of a vertical bar magnet near, but not touching,
the top of the oscilloscope cast just above the spot. Record the direction
of any deflection. Repeat with the S-pole.\vspace{20mm}

\item Turn off the oscilloscope.
\item Did the directions of deflections meet your expectations? Explain. \vspace{15mm}

\end{enumerate}
\textbf{Activity 2: A Moving Magnet and a Coil}

\begin{enumerate}
\item Turn on the computer and launch {\it EM Induction} in the 132 Workshop in the {\bf Start} menu.
\item Move the large coil as far as possible away from the small one.
\item Place a bar magnet vertically along the axis of the small coil with
the N-pole touching the coil.
\item Start recording data and lift the bar magnet quickly straight up.
\item At the end of the data taking interval, the computer should display
a value for the electromotive force (emf) induced in the small coil.
Several trials may be required to get the correct timing between starting
the data acquisition and removing the magnet. Note and record the sign of the induced
emf.\vspace{10mm}

\item \textbf{Prediction}: If you lower the magnet, N-pole down, quickly
toward the coil, what will be the sign of the emf? \vspace{15mm}

\item Carry out the experiment, starting the data acquisition, then lowering the magnet.
Record the sign of the induced emf.\vspace{10mm}

\item Did your result confirm or refute your prediction?\vspace{15mm}

\item \textbf{Prediction}: What will happen to the emf if you perform the
same pair of experiments with the S-pole toward the coil? \vspace{15mm}

\item Perform the two experiments, lifting and lowering the magnet, with
the S-pole down. Record the sign of the induced emf in each case.\vspace{10mm}

\item How did the results compare with your predictions?\vspace{15mm}

\end{enumerate}
\textbf{Activity 3: Two Coils and a Varying Current}

\begin{enumerate}
\item With the large coil laying flat on the table, place the small coil
at its center so that the axes of the coils coincide. Be sure no magnets
are near either coil.
\item Set the signal generator FREQUENCY knob to the 2K-20K position and
the frequency dial to 40. In other words, set the frequency to 4 kHz.
Set the function switch to SINE, the OUTPUT (VOLT P-P) RANGE knob
at 10 and the LEVEL knob at mid-range. Turn on the generator.
\item Run the VI. Two sine curves will be displayed: the red one is the
voltage signal being supplied to the large coil by the generator.
The blue curve is the emf induced in the small coil by the varying
current in the large coil. Note the amplitude of the induced current
and the phase relationship between the two curves. Ypu may have to
adjust the number of scans and scan rate to get good traces of the
signals.
\item Tilt the small coil to $45^\circ$ with the vertical, keeping the coil
at the center of the large coil. While holding the small coil in this
position, run the VI and note the amplitude of the induced current
and the phase relationship between the curves.
\item Repeat for rotations of $90^\circ$ and $180^\circ$, again noting the
amplitude and phase relationship. 
\item What can you conclude about the orientation of the secondary relative
to the primary in terms of the response of the secondary to a varying
current in the primary? Can you explain this? \vspace{15mm}

\item Move the coil to different positions inside and outside the large
coil. Run the VI for each position and note changes in amplitude and
phase of the induced emf in the small coil. \vspace{15mm}

\item Do your results support your explanation given in the previous question?
Justify your response.\vspace{15mm}
\end{enumerate}

